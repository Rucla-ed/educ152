
%external style tex file
%Karina's directory
\input{C:/Users/ozanj/Dropbox/econometrics-course/lectures/header-tex/hed612-lecture-style3} 
%\usepackage{multicol}

%Ozan's directory- need to change the style tex file to reflect UCLA email, course number, logos, etc.
%\input{C:/Users/ozanj/Documents/Dropbox/hed612/lectures/beamer-style/hed612-lecture-style2}


%\title [Short Title]{Long Title}
\title[EDUC 263, Lecture 4] {EDUC 263: Introduction to Econometrics, Lecture 4}
\subtitle{Introduction to matching (part 1 of matching unit)}
\date{Oct 23, 2017}

\begin{document}

\begin{frame}{Do this at beginning of class}

\vspace{3mm}	
	\begin{itemize}
	
		\item Download lecture 4 Stata ``do-file'' \href{https://www.dropbox.com/s/1hhixp495qrh0ws/intro-econometrics-lecture-4-do-file.do?dl=0}{\beamergotobutton{link}}
		\begin{itemize}
			\item Save to a folder you can easily find
			\item Note: you will have to change filepaths (``cd'' command) for do-file to run
		\end{itemize}
	
		\item Dataset we will be working with today
		\begin{itemize}
			\item ELS 2002 dataset [different than last week] \href{https://www.dropbox.com/s/ihjk5g54udjx78e/effect_cc_on_ba_dataset.dta?dl=0}{\beamergotobutton{link}}
			\item Save to folder you can easily find; do not change file names			

		\end{itemize}
		
	\end{itemize}
\end{frame}

	
\begin{frame}
	\titlepage
\end{frame}

\begin{frame}{Goals for three-week unit on matching}


	Today: Conceptual introduction to matching
		\begin{itemize}
			\item Gain deeper understanding of Rubin's Causal Model/ potential outcomes framework, which is core to matching		
			\item Introduce the parts of matching, so that you can understand matching as a whole
			
		\end{itemize}
	\vspace{3mm}		
	Following two weeks: Deeper understanding of matching
		\begin{itemize}
			\item Next week:
			\begin{itemize}
				\item Brief overview of logistic regression
				\item In-depth discussion of matching assumptions
				\item In-depth discussion of implementation and alternative matching methods
			\end{itemize}
			\item Following week:
			\begin{itemize}
				\item Inverse probability weighting: preferred matching approach
				\item More practice conducting matching
			\end{itemize}
		\end{itemize}

\end{frame}

\begin{frame}{What we will do today}
	%\textbf{Today}
	%\vspace{2mm}
	\tableofcontents
\end{frame}

\section[Introduce RQ]{Introduce research question: What is the effect of attending community college on BA attainment}

\begin{frame}{Research question we will use for matching unit}

	Research question:
	\begin{itemize}
		\item What is the effect of attending a community college (rather than 4-year institution) on BA attainment (Y)?
		%\item In some examples, we will use salary as Y
	\end{itemize}
	\vspace{2mm}
	
	Treatment variable $ T_i $
	\begin{itemize}
		\item $ T_i=1 $: first institution is public community college (CC)
		\item $ T_i=0 $: first institution is four-year institution			
	\end{itemize}
	\vspace{2mm}
	Data: Education longitudinal study of 2002 (ELS)
	\begin{itemize}
		\item 10th graders in 2002
	\end{itemize}
	\vspace{2mm}
	Sample:
	\begin{itemize}
		\item 12th grade in 2004; expect to obtain BA; attend a CC or 4-year by 2006
	\end{itemize}
	\vspace{2mm}	
	Run code together in Stata
\end{frame}

\section[RCM \& treatment effects]{Rubin's Causal Model (RCM) and treatment effects}

\subsection{Potential outcomes}

\begin{frame}{Review Rubin causal model (RCM) of potential outcomes}

	RQ: What is the effect of attending a community college ($T_i=1$) vs. 4-year ($T_i=0$) on salary (Y)?
	\begin{itemize}
		\item sample=high school grad in 2004; salary measured in 2011
	\end{itemize}
	\vspace{2mm}
	Each person $i$ has two \textbf{potential outcomes} \(Y_i\)
	\begin{itemize}
		\item \(Y_i(1)\) salary when assigned to treatment (community college)	
		\item \(Y_i(0)\) salary when assigned to control (4-year institution)
	\end{itemize}
	\vspace{2mm}
	How to think about potential outcomes
	\begin{itemize}
		\item for each person $ i$, the treated potential outcome $ Y_i(1) $ and the untreated potential outcome $ Y_i(0) $ already exist
		\item Treatment variable $ T_i $ just determines which of the two potential outcomes we get to observe	
	\end{itemize}	
\end{frame}

\begin{frame}{Review potential outcomes notation}{What is effect of attending community college (CC) vs. 4-year on salary (Y)?}

	\begin{itemize}
		\item Treatment, $T_i$
		\begin{itemize}
			\item 1=treated (CC); 0= untreated/control (4-year)
		\end{itemize}
		\vspace{2mm}		
		\item Four combinations of potential outcome and treatment assignment:
		\begin{itemize}
			\item \(Y_i(1) \mid T_i=1\) \ding{225} salary after treatment for those actually assigned to treatment (\textbf{observed})
			\item \(Y_i(0) \mid T_i=1\) \ding{225} salary after no treatment for those actually assigned to treatment (\textbf{not observed})
			\item \(Y_i(1) \mid T_i=0 \) \ding{225} salary after treatment for those assigned to control (\textbf{not observed})			
			\item \(Y_i(0) \mid T_i=0 \) \ding{225} salary after no treatment for those assigned to control (\textbf{observed})

		\end{itemize}
	\end{itemize}	
\end{frame}

\begin{frame}{Review potential outcomes notation}{What is effect of attending community college (CC) vs. 4-year on salary (Y)?}
	\begin{tabular}{r|c|c|}
	\multicolumn{1}{r}{}
		 					&  \multicolumn{1}{c}{Actually treated (T=1)} & \multicolumn{1}{c}{Actually not treated (T=0)} \\
		\cline{2-3}
		\(Y_i(1)\) & \textcolor{blue}{ \(Y_i(1) \mid T_i=1\)} & \textcolor{red}{\(Y_i(1) \mid T_i=0\)}  \\
		\cline{2-3}
		\(Y_i(0)\) & \textcolor{red}{\(Y_i(0) \mid T_i=1\)} & \textcolor{blue}{\(Y_i(0) \mid T_i=0\)}  \\
		\cline{2-3}
	\end{tabular}
	\\
	\vspace{5 mm}
	
	The two blue cells show the notation for the \emph{actual} outcomes
	\begin{itemize}
	\item E.g., for those who were actually treated (attended CC), their salary in 2011 is represented by \(Y_i(1) \mid T_i=1\)
	\item ``The treated potential outcome, for those who underwent treatment''
	\end{itemize}
	\vspace{3 mm}	
	Red cells are \emph{counterfactual} outcomes, which we never see
	\begin{itemize}
		\item Need counterfactual to estimate ``true'' causal effect
		\item Need some substitute for the counterfactual to estimate causal effect
	\end{itemize}
\end{frame}


\begin{frame}[shrink=10]{Table of potential outcomes}{What is effect of attending community college (CC) vs. 4-year on salary (Y)?}

	Imagine we know treated $ Y_i(1) $ and untreated $ Y_i(0) $ potential outcomes for all $i$ \\ 
	\vspace{3mm}
	\begin{tabular}{ l r r r}
		\multirow{2}{*} & $ Y_i(1) $ & $ Y_i(0) $ & $ \tau_i $ \\ \textbf{$ i $} & \textbf{Treated} & \textbf{Untreated} & \textbf{Unit effect} \\ \hline
		1 & 65 & 60 & 5 \\
		2 & 30 & 35 & -5 \\
		3 & 55 & 60 & -5 \\
		4 & 25 & 30 & -5 \\
		5 & 50 & 50 & 0 \\
		6 & 80 & 70 & 10 \\										
		7 & 45 & 45 & 0 \\ \hline
		\textbf{Avg} & \textbf{50} & \textbf{50} & \textbf{0} \\				
	\end{tabular}	
	
	\vspace{3mm}
	
	True causal effect for each $i$
	\begin{itemize}
		\item Unit treatment effect: for each person, compare their treated salary to untreated salary
		\item $UTE_i=Y_i(1)-Y_i(0)$

	\end{itemize}			
\end{frame}

\subsection{Treatment effects}

\begin{frame}{Types of treatment effects}{}

	\begin{enumerate}
		\item Unit treatment effect (UTE)
		\begin{itemize}
			\item For each person $i$, compare their treated to untreated $Y$
			\item $UTE_i=Y_i(1)-Y_i(0)$
			%\item Problem: we only observe one potential outcome for each $i$
		\end{itemize}			
		
		\item Average treatment effect (ATE)
		\begin{itemize}
			\item For all people ($T_i=1$ and $T_i=0$), what is the average effect of receiving treatment rather control?
			\item $ATE=E[Y_i(1)] - E[Y_i(0)] $
			%\item Problem: don't know $E[Y_i(1)]$ because we don't observe $Y_i(1)$ for everyone; don't know $E[Y_i(0)]$ because we don't observe $Y_i(0)$ for everyone
		\end{itemize}
		\item Average treatment effect on the treated (ATT)
		\begin{itemize}
			\item For people assigned to treatment ($T_i=1$), what is the average effect of receiving treatment rather control?
			\item $ ATT=E(Y_i(1)|T=1)-E(Y_i(0)|T=1) $
			%\item Problem: don't know $E(Y_i(0)|T=1) $ because don't observe untreated potential outcome for i assigned to treatment
		\end{itemize}
	\end{enumerate}	
\end{frame}

\begin{frame}[shrink=10]{Table of potential outcomes, calculating ATE}{What is effect of attending community college (CC) vs. 4-year on salary (Y)?}

	%Imagine we know $ Y_i(1) $ and $ Y_i(0) $ for all $i$ \\ 
	%\vspace{3mm}
	\begin{tabular}{ l r r r r}
		\multirow{2}{*} & $T_i$ & $ Y_i(1) $ & $ Y_i(0) $ & $ \tau_i $ \\ \textbf{$ i $} & \textbf{Assignment} & \textbf{Treated} & \textbf{Untreated} & \textbf{Unit effect} \\ \hline
		1 & 1 & 65 & 60 & 5 \\
		2 & 0 & 30 & 35 & -5 \\
		3 & 1 & 55 & 60 & -5 \\
		4 & 0 & 25 & 30 & -5 \\
		5 & 1 & 50 & 50 & 0 \\
		6 & 0 & 80 & 70 & 10 \\										
		7 & 1 & 45 & 45 & 0 \\ \hline
		\textbf{Avg} & & \textbf{50} & \textbf{50} & \textbf{0} \\				
	\end{tabular}	
	
	\vspace{3mm}
	
	Calculating average treatment effect (ATE)
	\begin{itemize}
		\item $ATE=  E[Y_i(1)] - E[Y_i(0)] =$
		\vspace{3mm}
		\item $ATE=\frac{1}{N} \sum_{i=1}^{N} Y_i(1) - \frac{1}{N} \sum_{i=1}^{N} Y_i(0) $
		\vspace{3mm}
		\item $ATE= 50-50=0$
	\end{itemize}			
\end{frame}

\begin{frame}[shrink=10]{Table of potential outcomes, calculating ATT}{What is effect of attending community college (CC) vs. 4-year on salary (Y)?}

	%Imagine we know $ Y_i(1) $ and $ Y_i(0) $ for all $i$ \\ 
	%\vspace{3mm}
	\begin{tabular}{ l r r r r}
		\multirow{2}{*} & $T_i$ & $ Y_i(1) $ & $ Y_i(0) $ & $ \tau_i $ \\ \textbf{$ i $} & \textbf{Assignment} & \textbf{Treated} & \textbf{Untreated} & \textbf{Unit effect} \\ \hline
		\textbf{1} & \textbf{1} & \textbf{65} & \textbf{60} & \textbf{5} \\
		2 & 0 & 30 & 35 & -5 \\
		\textbf{3} & \textbf{1} & \textbf{55} & \textbf{60} & \textbf{-5} \\
		4 & 0 & 25 & 30 & -5 \\
		\textbf{5} & \textbf{1} & \textbf{50} & \textbf{50} & \textbf{0} \\
		6 & 0 & 80 & 70 & 10 \\										
		\textbf{7} & \textbf{1} & \textbf{45} & \textbf{45} & \textbf{0} \\ \hline
		\textbf{Avg [treated]} & & \textbf{53.75} & \textbf{53.75} & \textbf{0} \\				
	\end{tabular}	
	
	\vspace{3mm}
	
	Calculating average treatment effect on the treated (ATT)
	\begin{itemize}
		\item $ATT=  E[Y_i(1)|T_i=1] - E[Y_i(0)|T_i=1] =$
		%\vspace{3mm}
		%\item $ATT=\frac{1}{N} \sum_{i=1}^{N} Y_i(1) - \frac{1}{N} \sum_{i=1}^{N} Y_i(0) $
		\vspace{3mm}
		\item $ATE= 53.75-53.75=0$
	\end{itemize}			
\end{frame}


\begin{frame}{Should you prefer average treatment effect (ATE) or average treatment effect on treated (ATT?)}{}

	\begin{itemize}
		\item ATE: for all people, what is the average effect of receiving treatment rather control? 
		\item ATT: for people assigned to treatment ($T_i=1$), what is the average effect of receiving treatment rather control? 
	\end{itemize}
	\vspace{3mm}
	Conceptually, depends and on policy debate/implications around your research question (RQ)
	\begin{itemize}
		\item RQ: effect of CC vs. 4-year on BA; policy debate about whether CC students would be better served at 4-year, suggests ATT
		\item RQ: effect of MAS on graduation; interested in whether all students (not just MAS participants ) would benefit from MAS, suggests ATE
	\end{itemize}

\end{frame}

\begin{frame}{Should you prefer average treatment effect (ATE) or average treatment effect on treated (ATT?)}{}

	\begin{itemize}
		\item ATE: for all people, what is the average effect of receiving treatment rather control? 
		\item ATT: for people assigned to treatment ($T_i=1$), what is the average effect of receiving treatment rather control? 
	\end{itemize}
	\vspace{3mm}

	Methodological considerations
	\begin{itemize}
		\item ATT requires fewer assumptions (next slide)
		\item If participation voluntary, treated likely differ from untreated in ways difficult to measure.
		\begin{itemize}
			\item Cleaner to estimate ATT: 
			\item ATE requires you to estimate effect of treated getting treatment rather than control (the ATT) \textbf{and} of untreated getting treatment rather than control
		\end{itemize}
	\end{itemize}		
\end{frame}

\begin{frame}{Estimating treatment effects is a missing data problem; we don't know counterfactual(s)}{}

	\begin{itemize}
		\item Unit treatment effect ($UTE_i=Y_i(1)-Y_i(0)$)
		\begin{itemize}
			%\item For each person $i$, compare their treated to untreated $Y$
			\item Missing data: we observe one potential outcome for each $i$
		\end{itemize}			
		
		\item Average treatment effect (ATE)
		\begin{itemize}
			\item For all people ($T_i=1$ and $T_i=0$), what is the average effect of receiving treatment rather control?
			\item $ATE=E[Y_i(1)] - E[Y_i(0)] $
			\item Two pieces of missing data:
			\begin{enumerate}
				\item Don't know $E[Y_i(1)]$ because don't observe $Y_i(1)$ for all $i$
				\item Don't know $E[Y_i(0)]$ because don't observe $Y_i(0)$ for all $i$
			\end{enumerate} 
		\end{itemize}
		\item Average treatment effect on the treated (ATT)
		\begin{itemize}
			\item For people assigned to treatment ($T_i=1$), what is the average effect of receiving treatment rather control?
			\item $ ATT=E(Y_i(1)|T=1)-E(Y_i(0)|T=1) $
			\item Only one piece of missing data:
			\begin{itemize}
				\item We know $E(Y_i(1)|T=1)$
				\item Don't know $E(Y_i(0)|T=1) $ because don't observe $Y_i(0) $  for $i$ assigned to treatment
			\end{itemize}
			
		\end{itemize}
	\end{itemize}	
\end{frame}

\begin{frame}{Random assignment experiments solve the missing data problem}{Average treatment effect on the treated (ATT)}

	\begin{itemize}
		\item Average treatment effect on the treated (ATT)
		\begin{itemize}
			\item For people assigned to treatment ($T_i=1$), what is the average effect of receiving treatment rather control?
			\item $ ATT=E(Y_i(1)|T=1)-E(Y_i(0)|T=1) $
			\item One piece of missing data:
			\begin{itemize}
				\item We know $E(Y_i(1)|T=1)$
				\item Don't know $E(Y_i(0)|T=1) $ because don't observe $Y_i(0) $  for $i$ assigned to treatment
			\end{itemize}
		\end{itemize}
	\item If value of $T_i$ randomly assigned:
	\begin{itemize}
		\item Treatment assignment unrelated to values of potential outcomes $Y_i(1) $ and $Y_i(0)$
		\item Can substitute in known quantity $E[Y_i(0)|T_i=0]$ for missing counterfactual $E[Y_i(0)|T_i=1]$
		\item $E[Y_i(0)|T_i=0]=E[Y_i(0)|T_i=1]$
		\item  $ATT=E[Y_i(1)|T_i=1] - E[Y_i(0)|T_i=0] $
	\end{itemize}
		
	\end{itemize}
	
\end{frame}


\begin{frame}{Random assignment solve missing data problem}{Average treatment effects (ATE)}

	\begin{itemize}
	
		\item Average treatment effect ($ATE=E[Y_i(1)] - E[Y_i(0)] $)
		\begin{itemize}
			\item ATE: For all people, what is the average effect of receiving treatment rather control?
			\item Two pieces of missing data:
			\begin{enumerate}
				\item Don't know $E[Y_i(1)]$ because don't observe $Y_i(1)$ for all $i$
				\item Don't know $E[Y_i(0)]$ because don't observe $Y_i(0)$ for all $i$
			\end{enumerate} 
		\end{itemize}
	\item If value of $T_i$ randomly assigned:
	\begin{itemize}
		%\item Treatment assignment unrelated to potential outcomes
		\item Substitute known quantities for missing counterfactuals
		%\item $E[Y_i(1)|T_i=1]=E[Y_i(1)|T_i=1]=E[Y_i(1)]$
		%\item $E[Y_i(0)|T_i=0]=E[Y_i(0)]$
	\end{itemize}
	\item Formula for $ATE $ more complicated than formula for $ATT $
	\begin{itemize}
		\item Must substitute for two counterfactuals
		\item Formula must include $Pr(T_i=1)$ and $Pr(T_i=0)$
	\end{itemize}
	\item ATE formula [don't worry about this]:
	\begin{itemize}
		%\item \scriptsize{$E(Y_i(1))= Pr(T=1)E(Y_i(1)|T=1)+Pr(T=0)E(Y_i(1)|T=0)$
		%\item $E(Y_i(1))=[Pr(T=1)E(Y_i(0)|T=1)+Pr(T=0)E(Y_i(0)|T=0)]$}
		\item \scriptsize{$ATE=E(Y_i(1))-E(Y_i(0))$
		\item $ATE=[Pr(T=1)E(Y_i(1)|T=1)+Pr(T=0)E(Y_i(1)|T=0)] - [Pr(T=1)E(Y_i(0)|T=1)+Pr(T=0)E(Y_i(0)|T=0)]$
		\item $ATE=Pr(T=1)[E(Y_i(1)|T=1)-E(Y_i(0)|T=1)]+Pr(T=0)[E(Y_i(1)|T=0)-E(Y_i(0)|T=0)]$
		}
	\end{itemize}
	
	\end{itemize}

	
\end{frame}


\section[Matching overview]{Overview of matching}

\begin{frame}{Estimating treatment effects when treatment not randomly assigned}{}

	\begin{itemize}
		\item Average treatment effect on the treated (ATT)
		\begin{itemize}
			\item For people assigned to treatment, what is the average effect of receiving treatment rather control?
			\item $ ATT=E(Y_i(1)|T=1)-E(Y_i(0)|T=1) $
			\item Problem: we don't know counterfactual $E(Y_i(0)|T=1) $
			\item Experiments substitute $E(Y_i(0)|T=0) $ for $E(Y_i(0)|T=1) $
		\end{itemize}
		\item If value of $T_i$ not randomly assigned:
		\begin{itemize}
			\item Treatment assignment likely related to values of potential outcomes $Y_i(1) $ and $Y_i(0)$
			\item $E(Y_i(0)|T=0) $ not a substitute for $E(Y_i(0)|T=1) $
		\end{itemize}
		\item Matching tries to replicate random assignment
		\begin{itemize}
			\item find untreated units that look just like treated units
			\item Assume that outcome for these untreated units $(Y_i(0)|T=0)$ is the counterfactual for treated units $(Y_i(0)|T=1)$
		\end{itemize}
	\end{itemize}
	

\end{frame}

\begin{frame}[shrink=10]{Conceptually, matching is like estimating causal effects when you know both potential outcomes}{}

	Imagine that bolded $i$ actually receive the treatment, $T_i=1$ \\
	
	\vspace{3mm}
	\begin{tabular}{ l r r r r}
		\multirow{2}{*} & $T_i$ & $ Y_i(1) $ & $ Y_i(0) $ & $ \tau_i $ \\ \textbf{$ i $} & \textbf{Assignment} & \textbf{Treated} & \textbf{Untreated} & \textbf{Unit effect} \\ \hline
		\textbf{1} & 1 & \textbf{65} & ? & ? \\
		2 & 0 & ? & \textbf{35} & ? \\
		3 & 0 & ? & 60 & ? \\
		\textbf{4} & 1 & \textbf{25} & ? & ? \\
		5 & 0 & ? & \textbf{50} & ? \\
		\textbf{6} & 1 & \textbf{80} & ? & ? \\										
		7 & 0 & ? & \textbf{45} & ? \\ \hline
		%\textbf{Avg} & \textbf{50} & \textbf{50} & \textbf{0} \\				
	\end{tabular}	
	
	\vspace{3mm}
	
	What matching does, Using $i=1$ as example
	\begin{itemize}
		\item $T_i=1 $ for $ i=1$
		\item actual: $(Y_i(1)|T_i=1)=65$
		\item counterfactual: $(Y_i(0)|T_i=1)=?$
		\item In order to estimate unit treatment effect, we need counterfactual
		\item Matching: choose an untreated observation that is similar to $i=1$; substitute $(Y_i(0)|T_i=0)$ for the counterfactual
	\end{itemize}
	\vspace{3mm}
	To calculate ATT
	\begin{itemize}
		\item Follow same steps for $i=4$, $i=6$
		\item ATT is avg of unit treatment effects from $i=1, i=4, i=6$
	\end{itemize}
	
\end{frame}

\begin{frame}[shrink=10]{So what is matching?}

Think of matching as \textbf{data preprocessing} rather than statistical analysis \\
\vspace{3mm}

Matching [average effect of treatment on treated (ATT)]
\begin{itemize}
	\item Goal: for each treated unit, we have $Y_i(1)|T=1$ but we need to find counterfactual $Y_i(0)|T=1$
	\item Matching identifies untreated units that are similar to each treated unit
	\item Use $Y_i(0)|T=0$ for the untreated unit as a substitute for $Y_i(0)|T=1$ of the treated unit
	\begin{itemize}
		\item Substitute for counterfactual may contain data from several untreated $i$s
	\end{itemize}
	%\item Find match for each treated unit; 
	\item For each treated unit, calculate the difference between actual  $Y_i(1)|T=1$ and substitute for the counterfactual
	\item ATT is just the average of this difference
\end{itemize}

\vspace{3mm}

Matching to calculate average treatment effect ATE
\begin{enumerate}
	\item Calculate ATT (as above)
	\item For each untreated $i$, find similar treated $i$ and follow steps above
\end{enumerate}

\end{frame}

%**********
\begin{frame}{Crucial assumption underlying matching}

	%\begin{quote}
	%[PSA] should only be applied if the underlying identifying assumption can be credibly invoked based on the informational richness of the data and a detailed understanding of the institutional set-up by which selection into treatment takes place ... Caliendo \& Kopeinig (2008), p. 32
	%\end{quote}
	
	Matching estimates causal effects [ATT] by
	\begin{enumerate}
		\item identify untreated $i$ that is \textbf{``similar''} to each treated $i$
		\item substituting $Y_i(0)|T=0$ for the untreated unit for the treated unit counterfactual $Y_i(0)|T=1$
	\end{enumerate}
	
	\vspace{2mm}
	
	What does it mean for untreated and treated $i$ to be similar?
	\begin{itemize}
		\item The matched treated and untreated $i$s have the same values on variables that determine the outcome and selection into the treatment 
	\end{itemize}
	
	\vspace{2mm}
	Critical assumption: conditional independence
	\begin{itemize}
		\item \textcolor{red}{When identifying untreated i similar to treated i, you include all variables that drive both the outcome \textbf{and} selection into the treatment}
		\item This is the same assumption underlying multiple regression (no omitted variables)	
	\end{itemize}


\end{frame}


\subsection[Matching vs. OLS]{Comparing matching to OLS}

\begin{frame}[shrink=10]{Difference between matching and OLS regression}{How regression estimates causal effects}

	Ordinary least squares (OLS) regression
	\begin{itemize}
		\item Population regression: $Y_i=\alpha T_i+\beta X_i + \epsilon_i$
		\begin{itemize}
			\item $\alpha$ and $\beta$ are \textbf{population parameters}
			\item $\alpha$ is ATE: average effect (for all) of receiving treatment rather than control
			\item $T_i=$treatment; $X_i=$ ``control'' variables
		\end{itemize}
		\item OLS regression line: $\hat{Y}_i=\hat{\alpha} T_i+\hat{\beta} X_i$
		\begin{itemize}
			\item $\hat{\alpha}$ is our estimate of ATE
			\item Reynolds and DesJardins (2009) describe regression as ``parametric'' in that we are estimating parameters
		\end{itemize}
	\end{itemize}
	\vspace{2mm}
	Assumption to claim $\hat{\alpha}$ is unbiased estimate of $\alpha$ (the ATE):
	\begin{itemize}
		\item $E(u_{i}|T_{i}) = 0$ ; Independent variable $X_{i}$ is unrelated to the ``other factors", $u_{i}$, not included in model
		\item i.e., no omitted variables that both (i) affect $Y_i$ \textbf{and} (ii) systematically related to $T_i$
		\item Also assumes we correctly specified the functional form linking $Y_i$ to control variables $X_i$
		
	\end{itemize}
\end{frame}

\begin{frame}[shrink=10]{Difference between matching and OLS regression}{How regression estimates causal effects}

	Parametric approach to estimating treatment effects (e.g., OLS)
	\begin{itemize}
		\item Use regression to estimate the value of the missing counterfactual
		\item Statistically control for observable characteristics
	\end{itemize}
	\vspace{2mm}
	Matching approach
	\begin{itemize}
		\item Substitute for counterfactual is not a parameter estimate; rather, choose an observation that represents best counterfactual for each treated $i$
		\item Instead of controlling for characteristics, group together treated and untreated $i$s that have same value on observable characteristics 
	\end{itemize}
\end{frame}

\begin{frame}{Difference between matching and OLS regression}{How regression estimates causal effects}

	How regression constructs the counterfactual, $Y_i(0)|T_i=1$
	\begin{enumerate}
		\item Estimate OLS line:
		\begin{itemize}
			\item $\hat{Y}_i=\hat{\alpha} T_i+\hat{\beta} X_i$
		\end{itemize}
			\item Predicted outcome given treatment:
		\begin{itemize}
		\item $ \hat{Y}_{1i} = \hat{\alpha}(T=1)+ \hat{\beta}(X_i)= \hat{\alpha}+ \hat{\beta}(X_i)$
		\end{itemize}
		\item Estimate counterfactual ($Y_i(0)|T_i=1$) by T=1 with T=0
		\begin{itemize}
			\item $ \hat{Y}_{0i} = \hat{\alpha}(T=0)+ \hat{\beta}(X_i)= \hat{\beta}(X_i)$
		\end{itemize}
	\end{enumerate}
	\vspace{2mm}
	How matching constructs the counterfactual, $Y_i(0)|T_i=1$
	\begin{itemize}
		\item Matching methods do not use estimated parameters to construct counterfactual
		\item Missing counterfactual ($Y_i(0)|T_i=1$) constructed from actual outcomes of untreated units that are matched to treated units

	\end{itemize}
\end{frame}


\subsection[Matching by hand]{Simple example of matching by hand}

	% Go through Porter and/or Reynolds/DesJardins example

\begin{frame}{``Nearest neighbor'' matching example by hand}
%\subsection{By hand}

	No mystery with matching, we simply create a control group that looks like the treated
	\begin{itemize}
		\item So we could actually do this by hand
	\end{itemize}
	
	\vspace{3mm}
	
	Let's try! Do this exercise with person next to you:
	\begin{itemize}
		\item Open the ``NN matching example'' spreadsheet \href{https://www.dropbox.com/s/ajqmhzomdwaxzt0/NN\%20matching\%20example.xlsx?dl=0}{\beamergotobutton{link}}
		\item Fictitious data
		\item You identify matches and calculate ATE, ATT
		\item Instructions at bottom
	\end{itemize}

\end{frame}

\section[Curse and cure]{The propensity score as the cure for the curse of dimensionality}

\subsection[Curse]{The curse of dimensionality}

\begin{frame}[shrink=10]{Curse of dimensionality}

	Goal: match treated $i$ to ``similar'' untreated $i$
	\begin{itemize}
		\item ``Similar'' untreated units have same values on variables that drive both the outcome and selection into treatment
		\item Assumption: when identifying untreated i similar to treated i, you include \textbf{all} variables that drive both the outcome and selection into the treatment
	\end{itemize}
	
	\vspace{2mm}
	For each possible combination of $X$s, must have at least one untreated $i$ to match to treated $i$
	\begin{itemize}
		\item Not an issue when sample size is large and small number of variables drive $Y$ and selection into treatment, $T$
		\item For example, imagine ``first-gen'' is only X that affects both $Y$ and $T$
	\end{itemize}
	
	\vspace{3mm}
	\begin{table}
	\begin{tabular}{l | r | r}
		\hline & first-gen=1 & first-gen=0 \\ \hline
		Start at CC  ($T_i=1$) & 40 & 10 \\ \hline
		Start at 4-yr  ($T_i=0$) & 10 & 40 \\ \hline
	\end{tabular}
	\caption{Number of observations in each cell, selection on one $X$ variable}
	\end{table}

\end{frame}

\begin{frame}[shrink=10]{Curse of dimensionality}

	For each possible combination of $X$s, must have at least one untreated $i$ to match to treated $i$
	\begin{itemize}
		\item Often have empty cells as number of $X$s increases
		\item Below, there are no 4-yr students ($T_i=0$) to match with the 25 first-gen, low-income CC students ($T_i=1$)
	\end{itemize}
	
	%\vspace{3mm}
	
	\begin{table}
	\begin{tabular}{l | r r r | r r r}
		\hline & \multicolumn{3}{c}{first-gen=1}  & \multicolumn{3}{c}{first-gen=0} \\ 
		Family income & low & med & high & low & med & high \\ \hline	
		Start at CC ($T_i=1$) &  \textbf{25} & 10 & 5 & 5 & 5 & \textbf{0} \\ \hline
		Start at 4-yr ($T_i=0$) & \textbf{0} & 5 & 5 & 5 & 10 & \textbf{25} \\ \hline
	\end{tabular}
	\caption{Number of observations in each cell, selection on two $X$ variables}
	\end{table}
	
	%\vspace{3mm}
	
	Reyrnolds and DesJardins (2009, p. 59):
	{\scriptsize
		\begin{quote} ``To eliminate the selection problem, we need to match on a large number of characteristics, some of which may be continuous. Even if we categorize a continuous variable, we would like to make the categories as fine as possible to reduce selection bias. As X becomes larger in dimension there is a higher probability of missing cells, a problem known as the “curse of dimensionality.”''  \end{quote}
		}
			
\end{frame}

\subsection[Cure]{The propensity score}

\begin{frame}[shrink=10]{Propensity score cures the curse of dimensionality}
%\subsection{By hand}
	
	%$P(T=1)$
	%($\hat{P}(T=1|X)$)
	Rosenbaum and Rubin (1983) show that:
	\begin{itemize}
		\item Matching on predicted probability of treatment given X is equivalent to matching on X
		\item Matching on probability of treatment balances distribution of observable characteristics across treated and untreated groups
		\begin{itemize}
			\item Probability that any treated unit has specific trait (e.g., high income) will be same as probability untreated unit has that trait
		\end{itemize}
		
	\end{itemize}
	
	\vspace{3mm}
	Solution to the curse of dimensionality
	\begin{itemize}
		\item Propensity score ($\hat{P}_i(X)$): probability unit $i$ treated given their observable characteristics X
		\item Instead of matching on individual characteristics (X), match on probability of treatment given X
		\begin{itemize}
			\item e.g., match treated $i$ with ($\hat{P}_i(X)=.8$) to untreated $i$ with ($\hat{P}_i(X)=.8$)
		\end{itemize}
		\item Matching on propensity score reduces matching problem to a single dimension
	\end{itemize}

\end{frame}

\begin{frame}{How to calculate and utilize propensity score}
%\subsection{By hand}
	
	\begin{enumerate}
		\item Run regression to estimate probability of treatment
		\begin{itemize}
			\item Y= receipt of treatment (0/1)
			\item X=variables that affect outcome of interest and selection into treatment
			\item Usually, this is ``logistic'' or ``probit'' regression, rather than OLS
			\item $\hat{Y}_i$ from this regression is the propensity score (probability of treatment)
		\end{itemize}
		\item Match treated units with untreated units that have similar propensity scores
		\item For each treated $i$, calculate difference between $Y_i$ and $Y_i$ for matched untreated unit
		\item Calculate ATT as average of these differences
	\end{enumerate}
\end{frame}

\begin{frame}{Practical steps in propensity score analysis}{Not necessarily in this order}
%\subsection{By hand}
	
	\begin{enumerate}
		\item Decide which treatment effect to estimate: ATE, ATT?
		\item Identify Xs that drive Y and selection into treatment
		\begin{itemize}
			\item Assess assumption 1: conditional independence
		\end{itemize}
		\item Calculate propensity scores from logit/probit regression
		\item Choose method to match treated and untreated
		\begin{itemize}
			\item e.g., ``nearest neighbor''(NN); NN with caliper; kernal; inverse propensity weights
		\end{itemize}
		\item Assess assumptions 2 and 3 (will cover later)
		\begin{itemize}
			\item Assumption 2 (``common support''): for each treated $i$ is there untreated $i$ with same propensity?
			\item Assumption 3 (``balance''): after matching, are untreated similar to treated on observed characteristics?
		\end{itemize}
		\item Estimate causal effect (ATT, ATE)
	\end{enumerate}
\end{frame}

\section[Practical example]{Practical example of implementing matching using ``nearest neighbor'' approach}

\begin{frame}{Choose method (algorithm) to match controls to treated}
Generally, take a treated unit and find one or more control units with a similar propensity
\begin{itemize}
	\item Key: face a tradeoff between bias and efficiency (standard errors)
	\item Limit matches to very good matches (very similar propensities)
			\begin{itemize}
				\item End up with small sample size (high SEs) and little bias
			\end{itemize}
	\item Allow matches to differ
			\begin{itemize}
				\item Larger sample size (low SEs), but probably some bias, because matches are not as good
			\end{itemize}
\end{itemize}
\vspace{3mm}

All depends on your particular application, the number of treated and control units, and the amount of common support
\end{frame}


%**********
\begin{frame}{Nearest neighbor method to match controls to treated}
%\subsection{NN}
\begin{itemize}
	\item ``Nearest neighbor''
	\begin{itemize}
		\item Pick a treated unit, find control whose propensity score is closest
	\end{itemize}
	\item With or without replacement
		\begin{itemize}
			\item Without replacement: once a treated and control unit are matched, the control is pulled out of the pool
			\item With replacement: single control can be matched to multiple treated units 
			\item Tradeoff between bias and efficiency
		\end{itemize}
	\item Can also oversample (more than one control per treated); recommended if your sample is larger enough
	\item Generally not used by itself
			\begin{itemize}
				\item Nearest neighbor could be very far away (i.e., a bad match)
				\item Use of a caliper solves this problem
			\end{itemize}
\end{itemize}
\end{frame}



%**********
\begin{frame}{Nearest neighbor with caliper [SKIP/SKIM]}
What is a caliper in real life?

\vspace{3mm}

\begin{itemize}
	\item Define a caliper $c$, usually $c \le .25\sigma$, where $\sigma$ is calculated from the propensity score  
	\item Search for nearest neighbor only among C's within the the caliper
			\begin{itemize}
				\item Out of these C's, choose the nearest
			\end{itemize}
	\item Avoids bad matches, but may result in fewer matches
	\item Caliper width is atheoretical
			\begin{itemize}
				\item Should be convincing
				\item Consider a caliper of .01 versus .10
			\end{itemize}
\end{itemize}

\end{frame}



%**********
\begin{frame}{Assess match quality  [SKIP/SKIM]}
Remember: goal of propensity model and matching process is to yield two groups that look alike for a given set of covariates 
\vspace{3mm}

Several ways to see if this is the case
\begin{itemize}
	\item Difference in means
	\item Standardized bias - difference between sample means as a percentage of square root of variances
	\[ 100*\frac{\bar{X}_T-\bar{X}_C}{\sqrt{.5(V(X_C)+V(X_C))}}  \]
\end{itemize}

\vspace{3mm}

Can use tables or figures to display information
\begin{itemize}
	\item For many covariates, plot distribution of bias before and after
\end{itemize}

\end{frame}


%**********
\begin{comment}
\begin{frame}{Steps for a NN analysis with caliper}
\begin{itemize}
	\item Develop a propensity model and estimate the propensities
	\item Check for common support
	\item Create the size of the caliper
	\item Match using the caliper
	\item Check that means for your covariates are equivalent for treated and controls 
	\item Test for differences between the two groups ($t$-test or multivariate)
	\item Ask whether your results hold qualitatively for population
\end{itemize}
\end{frame}

\end{comment}

%**********
\begin{frame}{Practical example: What is the effect of mother being a smoker (X) on birth weight (Y)?}
%\subsection{psmatch2}

	Work through example in lecture 4 do-file:
	\begin{itemize}
		\item Will have to install some commands
		\item Dataset:
		\begin{itemize}
			\item 4,642 observations
			\item Each observation represents a mother
		\end{itemize}
	\end{itemize}
		
	\begin{table}[htbp]
	  \centering
	  \scalebox{.9}{
	    \begin{tabular}{lcccr}
	   \hline
	          & Full sample & Smokers & Non-smokers & \multicolumn{1}{c}{$\Delta$} \\
	    \hline
	    Prenatal visit 1st tri. & .80  & .69  & 0.83  & -.14 \\
	    Married  & .70  & .47  & 0.75  & -.28 \\
	    Mother's age & 26.5  & 25.2  & 26.8  & -1.60 \\
	    Mother's education & 12.70 & 11.60 & 12.90 & -1.30 \\ \hline
	    n & 4,642 & 864 & 3,778 \\
	    \hline
	    \end{tabular}%
	    }
	\end{table}%

	\vspace{3mm} 
	
	Mothers who smoke are less likely to see the doctor, be married, are younger, and have less education.


\end{frame}


	%curse of dimensionality
	%WHAT TO DO NEXT: EXAMPLE OF NEAREST NEIGHBOR IN STATA? DETAILED DISCUSSION OF ASSUMPTIONS? INTRODUCTION TO LOGISTIC REGRESSION?

\end{document}

