
%external style tex file
%Karina's directory
\input{C:/Users/ozanj/Dropbox/econometrics-course/lectures/header-tex/hed612-lecture-style3} 

%Ozan's directory- need to change the style tex file to reflect UCLA email, course number, logos, etc.
%\input{C:/Users/ozanj/Documents/Dropbox/hed612/lectures/beamer-style/hed612-lecture-style2}


%\title [Short Title]{Long Title}
\title[EDUC 263, Lecture 2] {EDUC 263: Introduction to Econometrics, Lecture 2}
\subtitle{Experimental and observational designs}
\date{Month Date, Year}

\begin{document}

\begin{frame}{Do this at beginning of class}
	\begin{itemize}

		\item Download lecture 2 Stata ``do-file'' \href{https://www.dropbox.com/s/qou0320ctms7gcw/intro-econometrics-lecture-2-do-file.do?dl=0}{\beamergotobutton{link}}
		\begin{itemize}
			\item Save to a folder you can easily find
		\end{itemize}
	
		\item Download lecture 2 datasets
		\begin{itemize}
			\item Save to folder you can easily find; do not change file names
			\item California school dataset \href{https://www.dropbox.com/s/zudnfqugke33onf/caschool-v2.dta?dl=0}{\beamergotobutton{link}}
			\item Tennessee STAR Experiment dataset  \href{https://www.dropbox.com/s/784ej11tn8q9ykk/STAR_Students.dta?dl=0}{\beamergotobutton{link}}
		\end{itemize}
		\vspace{4mm}		 
		\item Try doing this in Stata:
		\begin{enumerate}
			\item Open Stata
			\item Open ``do-file'' editor
			\item Open lecture 2 do-file
			\item ``change directories'' in do-file so that file-path in ``cd'' command points to where you saved the data
			\begin{itemize}
				\item You will change file-paths in two places in do-file: once for CA school data; once for Tennesee STAR data
			\end{itemize}
			
		\end{enumerate}
		
	\end{itemize}
\end{frame}

	
\begin{frame}
	\titlepage
\end{frame}

\begin{frame}{What we will do today}
	%\textbf{Today}
	%\vspace{2mm}
	\tableofcontents
\end{frame}

\section[Stata]{Introduction to Stata}

\begin{frame}{Understanding syntax of Stata commands}

	Let's open a dataset in Stata. The ``auto'' dataset is a sample dataset saved on your computer when you install Stata. \\
	\vspace{3mm}
	Type the following text in the Stata commandline and then press ``Enter'' on your keyboard: \\
	
	\begin{itemize}
		\item [] \texttt{sysuse auto, clear}
	\end{itemize}	
	
	Type the following text in the Stata commandline and then press ``Enter'' on your keyboard: \\
	
	\begin{itemize}
		\item [] \texttt{describe}
	\end{itemize}	

\end{frame}

%**********
\begin{frame}{Understanding syntax of Stata commands}
	
	Stata commands often follow the following format:
	
	\begin{itemize}
		\item [] \texttt{\underline{com}mandname [varlist] [if] [in] [, options]}	
		\begin{itemize}
			\item only need to type underlined part of command name
			\item anything in brackets doesn't need to be included for command to run
			\item  text after the comma are options
		\end{itemize}		
	\end{itemize}
	
	\vspace{3mm}
	For example, the summarize command:
	\begin{itemize}
		\item [] 	\texttt{\underline{su}mmarize [varlist] [if] [in] [weight] [, options]}
	\end{itemize}	
	
	\vspace{3mm}
	Try typing these commands in Stata command line:
	\begin{itemize}
		\item [] \texttt{summarize}
		\item [] \texttt{sum price}
		\item [] \texttt{sum price, detail}		
	\end{itemize}	
	
\end{frame}

\begin{frame}{Executing Stata commands}

	Three ways to execute Stata commands:	
	\vspace{2mm}	
	\begin{itemize}
		\item Point-and-click (Ugh!)
		\item Stata command line
		\begin{itemize}
			\item Will use this to run individual commands
		\end{itemize}			
		\item Stata do-file
		\begin{itemize}
			\item Best way to run Stata commands; required for all homework assignments
			\item Will review working with do-files later in this lecture
		\end{itemize}			
	\end{itemize}	
	
\end{frame}

\begin{frame}{Stata help files}
	
	In Stata, type following syntax:
	\begin{itemize}
		\item [] \texttt{help \underline{com}mandname}
		\item [] \texttt{help generate}
		\item [] \texttt{help gen}
		\item [] \texttt{help reg}
	\end{itemize}

\end{frame}

%**********
\begin{frame}[shrink=10]{Reading Stata help files}
	
	Help files may feel too technical at first, but you'll become more comfortable with practice

	\vspace{3mm}	
	Help files follow a standard outline (e.g., \texttt{help reg}): 
	\begin{itemize}
		\item Syntax (command syntax and list of options)
		\item Menu (how to execute command using point-and-click)
		\item Description (text overview of what command does)
		\item Options (detailed description of command options)
		\item Examples (examples of how to use command)
		\item Video example (some commands have this)
		\item Stored results (stored results created by command)
	\end{itemize}	
	\vspace{3mm}	
	Top right corner of help file:
	\begin{itemize}
		\item ``Dialog'' (run command using point and click)
		\item ``Also see'' (link to PDF documentation; related Stata commands)		
	\end{itemize}			

\end{frame}

%**********
\begin{frame}{Working with Stata ``do-files''}
	
	A Stata do-file is just a text-file that contains Stata commands \\
	\vspace{3mm}
	Opening a do-file:
	\begin{itemize}
		\item Open do-files from \textbf{within} Stata rather than from Windows Explorer (or Mac equivalent)
		\item In Stata, click on the ``\textcolor{red}{New do-file editor}'' button; this opens a new do-file
		\item In the do-file, click on ``\textcolor{red}{file}'' then ``\textcolor{red}{open}'' and the find the do-file you want to open
		%\item Let's open the do-file called \textcolor{red}{\texttt{hed616-lecture1-do-file1}}
	\end{itemize}
	
\end{frame}

%**********
\begin{frame}[shrink=3]{Executing Stata commands within a do-file}
	
	Within a do-file you can run commands several different ways (try doing this within the do-file):
	\begin{enumerate}
		\item One command at a time by highlighting only that command and clicking \textcolor{red}{Execute(do)} button in do-file
		\item Several commands at a time by highlighting several commands and clicking \textcolor{red}{Execute(do)} button in the do-file
		\item can run the entire do-file by not highlighting any commands and clicking \textcolor{red}{Execute(do)} button in do-file
	\end{enumerate}	
	
	\vspace{3mm}
	
	\textcolor{red}{Comments}: text that Stata will ignore when executing do-file
	\begin{itemize}	
		\item Different ways to start a comment:
		\begin{itemize}
			\item \texttt{* COMMENT}
			\item \texttt{// COMMENT}
			\item \texttt{/* COMMENT */}
		\end{itemize}
	\end{itemize}		
\end{frame}

%**********
\begin{frame}{Changing directories within a Stata do-file}
	
	``Working directory'' is the directory (i.e., filepath) where Stata looks to find files (e.g., datasets)

	\vspace{3mm}
	The \textcolor{red}{\texttt{cd}} command changes the filepath of the working directory. Syntax: \\
	\begin{itemize}
		\item [] \texttt{cd "filepath"}
		\begin{itemize}
			\item [] Note: PC uses backslash ``\textbackslash'' to separate folders in filepath
			\item [] Note: Mac uses forward-slash ``/'' to separate folders in filepath
		\end{itemize}
	\end{itemize}
	
	\vspace{3mm}	
	Essential that you become comfortable changing directories within do-files \\
	\vspace{3mm}
	Let's practice in the do-file	
	
\end{frame}

\section[Goal of statistics]{Goal of statistics}

\begin{frame}\frametitle{Goal of statistics}

	Goal of statistics (including econometrics):
	\bi
		\item use sample data to make statement about population
	\ei
	\vspace{2mm}	
	Population parameter: some measure of the population
	\bi
		\item e.g., mean income across all U.S. households
		\item Usually don't know this; need data on entire population
	\ei
	\vspace{2mm}	

	Estimator
	\bi
		\item A formula or procedure used to calculate an educated guess of the value of the population parameter
		\item e.g., calculating difference between treated and untreated mean in experiments; ordinary least squares (OLS) estimator
	\ei
	Point estimate (or estimate):
	\bi
		\item Numeric value calculated when you apply an estimator to a specific sample of data.
		\item e.g., calculate sample mean income is \$45,000
	\ei
	
\end{frame}

\begin{comment}
\begin{frame}
	\frametitle{Parameters and point estimates}
	
	
	In introductory statistics class, the primary parameter we were interested in was the population mean, $ \mu_Y $, and the estimator we used most often was the sample mean, $ \bar{Y} $ \\
	
	\vspace{3mm}
	In multivariate regression, primary parameter we are interested in is the population regression coefficient, $ \beta $, which we estimate with the sample regression coefficient, $ \hat{\beta} $
	\vspace{2mm}
	\bi
		\item RQ: what is the effect of living in residence halls on GPA?
		\item Population regression coefficient for living on campus would tell us the true causal effect of living on campus on GPA for the entire population
	\ei
	
	%$ Y_{it}=\beta X_{i,t-1} +W'_{i,t-1}\gamma+\delta_t+\alpha_i+\epsilon_{i,t} $
	
	
\end{frame}
\end{comment}

\begin{frame}
	\frametitle{Notation: parameters vs. estimates}
	
	
	Population parameters
	\bi
		\item Described using lowercase Greek letters (e.g., $ \mu, \sigma, \beta $)
		\bi
			\item e.g., $ \mu $ (``mu'') refers to population mean; $ \sigma $ (``sigma'') refers to population standard deviation
		\ei
		Subscripts usually denote variables (e.g., $ \mu_Y, \sigma_X, \beta_X $)
		\bi
			\item $ \sigma_X $ refers to population standard deviation of variable X
		\ei
	\ei
	\vspace{3mm}
	
	Estimates of population parameters
	\bi
		\item Described using Greek letters with ``hat''
		\bi
			\item e.g., $ \hat{\mu}_Y $ is the estimate of $ \mu_Y $
			\item $ \hat{\sigma}_X $ is estimate of $ \sigma_X $ based on sample data
		\ei
			\item Also described using Arabic letters
		\bi
			\item e.g., $ \bar{Y} $ is estimate of $ \mu_Y $ based on sample data
			\item $ s_X $ is estimate of $ \sigma_X $ based on sample data
		\ei
	\ei

\end{frame}

\subsection[Bias \& Efficiency]{Bias \& Efficiency}

\begin{frame}{Desirable properties of estimators: Efficiency and unbiasedness}
	\begin{itemize}
	\item Desirable properties of your point estimates (e.g., $\hat{\beta}$ or $\overline{Y}$)
		\begin{itemize}
		\item Desire point estimates to be ``unbiased"
		\item Desire point estimates to be ``efficient"
		\end{itemize}
	\item Efficiency 
		\begin{itemize}
		\item Definition:
			\begin{itemize}
			\item Efficiency refers to how close your point estimate is to the population parameter
			\end{itemize}
		\item Standard error:
			\begin{itemize}
			\item On average, how far away is a point estimate from one random sample from the value of the population parameter			
			\end{itemize}
		\item Therefore, an efficient point estimate is one with low standard error
		\end{itemize}	
	\end{itemize}
\end{frame} 

\begin{frame}[shrink=10]{Bias}
	\begin{itemize}
	\item An ``unbiased" point estimate
		\begin{itemize}
		\item A point estimate is ``unbiased" if value of point estimate gets closer to value of true population parameter as sample size increases
		\end{itemize}
	\item Bias
		\begin{itemize}
		\item Bias occurs when point estimate does not get closer to population parameter as sample size increases
		\item A biased estimate consistently overestimates or underestimates population parameter in repeated random samples
		\item There are many different types of bias
		\end{itemize}
	\item Sampling bias:
		\begin{itemize}
		\item The estimate of population parameter is biased because you fail to take a random sample
		\item Example: goal is to estimate high school graduation rate
			\begin{itemize}
			\item You take random sample of 10th grades and see if they graduate within three years
			\end{itemize}
		\end{itemize}
	\item Omitted variable bias: 
		\begin{itemize}
		\item Bias in estimate of $\beta$ due to omitting necessary ``control" variables from your regression model
		\end{itemize}
	\end{itemize}
\end{frame}

\begin{frame}{Unbiased estimates of causal relationships more difficult than descriptive relationships}
	\begin{itemize}
		\item Unbiased estimate of a population mean
		\begin{itemize}
			\item  e.g., what is mean hours worked per week in U.S.
			\item Primary threat is sampling bias; is your sample representative of the population
		\end{itemize}
		\item Unbiased estimate of a correlational relationship
		\begin{itemize}
			\item e.g., how much longer (shorter) do married men live compared to unmarried men?
			\item Primary threat is sampling bias		
		\end{itemize}
		\item Unbiased estimate of a causal relationship
		\begin{itemize}
			\item Effect of marriage (X) on life expectency (Y) for men?
			\item Threats to unbiased estimate
			\begin{itemize}
				\item Sample unrepresentative of population
				\item Mistaking a correlational relationship for a causal one; even if you had data on \textbf{entire population}, your estimate could be biased 
			\end{itemize}
		\end{itemize}
	\end{itemize}
\end{frame}


\section[Experiments]{Experiments}

\begin{frame}{Potential outcomes}

	Example: what is effect of having an internship in college (X) on earning after college (Y)?
	\begin{itemize}
		\item let $ i=1...N $ be units (e.g., people) in sample
		\item $ d_i $ indicates receipt of treatment (e.g., internship)
		\begin{itemize}
			\item $ d_i=1 $ for treated units; $ d_i=0 $ for untreated units
		\end{itemize}
		\item ``Potential outcomes'', $ Y_i(1) $ and $ Y_i(0) $
		\begin{itemize}
			\item $ Y_i(1) $: outcome if $ i $ if $ i $ receives treatment $ d_i=1 $
			\item $ Y_i(0) $: outcome if $ i $ if $ i $ doesn't receives treatment $ d_i=0 $
			%\item In real world, we only observe one outcome; missing outcome is the counterfactual			
			%\item In general, potential outcomes may be written $ Y_i(d ) $, where $ d $ indexes the treatment
		\end{itemize}
		\item ``Observed outcome,'' $ Y_i $
		\begin{itemize}
			\item For each person, we observe $ Y_i(1) $ or $ Y_i(0)$ but never both
			\item $ Y_i=Y_i(1)d_i+Y_i(0)(1-d_i)$
			\item if $ d_i=1 $ (treated):
			\begin{itemize}
				\item  $Y_i=Y_i(1)*1+Y_i(0)(1-1)=Y_i(1)$
				%\item The observed outcome, $ Y_i $, equals the potential outcome, $Y_i(1)$ if the treatment is administered ($d_i=1$)
			\end{itemize}
			\item if $ d_i=0 $ (untreated): 
			\begin{itemize}
				\item $Y_i=Y_i(1)*0+Y_i(0)(1-0)=Y_i(0)$
				%\item The observed outcome, $ Y_i $, equals the potential outcome, $Y_i(0)$ if the treatment is administered ($d_i=0$)
			\end{itemize}

		\end{itemize}

	\end{itemize}

\end{frame}


\begin{comment}
		\item distinction between $ d_i $ and $ D_i $ 
		\begin{itemize}
		\item $ d_i $ the treatment that a given subject receives (a variable that one observes in an actual dataset)
		\item $ D_i $ the treatment that could be administered hypothetically. 
		\begin{itemize}
			\item $ D_i $ is a random variable
			\item the $i$th subject might be treated in one hypothetical study and not another
			\item $ Y_i(1)|D_i=1 $ refers to the ``treated potential outcome for a subject who would be treated under some hypothetical allocation of treatments
		\end{itemize}
				
		\end{itemize}
\end{comment}

\begin{frame}{Table of potential outcomes}

	Example: what is effect of having an internship in college (X) on annual income after college (\$000s) (Y)? \\
	\vspace{3mm}
	\begin{tabular}{ l r r r}
		\multirow{2}{*} & $ Y_i(1) $ & $ Y_i(0) $ & $ \tau_i $ \\ \textbf{$ i $} & \textbf{Treated} & \textbf{Untreated} & \textbf{Treatment effect} \\ \hline
		1 & 65 & 60 & 5 \\
		2 & 30 & 35 & -5 \\
		3 & 55 & 60 & -5 \\
		4 & 25 & 30 & -5 \\
		5 & 50 & 50 & 0 \\
		6 & 80 & 70 & 10 \\										
		7 & 45 & 45 & 0 \\ \hline
		\textbf{Average} & \textbf{50} & \textbf{50} & \textbf{0} \\				
	\end{tabular}	
\end{frame}

\begin{frame}[shrink=10]{How to think about relationship between potential outcomes and observed outcome}

	%Example: what is effect of having an internship in college (X) on annual income after college (\$000s) (Y)?
	%\vspace{3mm}
	\begin{tabular}{ l r r r}
		\multirow{2}{*} & $ Y_i(1) $ & $ Y_i(0) $ & $ \tau_i $ \\ \textbf{$ i $} & \textbf{Treated} & \textbf{Untreated} & \textbf{Treatment effect} \\ \hline
		1 & 65 & 60 & 5 \\
		2 & 30 & 35 & -5 \\
		3 & 55 & 60 & -5 \\
		4 & 25 & 30 & -5 \\
		5 & 50 & 50 & 0 \\
		6 & 80 & 70 & 10 \\										
		7 & 45 & 45 & 0 \\ \hline
		\textbf{Average} & \textbf{50} & \textbf{50} & \textbf{0} \\				
	\end{tabular}
	\vspace{3mm}
	
	How to think about potential vs. observed outcomes:
	\begin{itemize}
		\item for each person $ i$, the treated potential outcome $ Y_i(1) $ and the untreated potential outcome $ Y_i(0) $ already exist
		\item Treatment $ d_i $ just determines which of the two potential outcomes we get to observe
		\item for each paerson, the only difference between $ Y_i(1) $ and $ Y_i(0) $ is the treatment
		\item Value of potential outcomes is driven by the treatment and by characteristics that affect $Y_i$ (e.g., parental income)
	\end{itemize}		
\end{frame}

\begin{frame}[shrink=10]{Average treatment effect (ATE)}

	\begin{tabular}{ l r r r}
		\multirow{2}{*} & $ Y_i(1) $ & $ Y_i(0) $ & $ \tau_i $ \\ \textbf{$ i $} & \textbf{Treated} & \textbf{Untreated} & \textbf{Treatment effect} \\ \hline
		1 & 65 & 60 & 5 \\
		2 & 30 & 35 & -5 \\
		3 & 55 & 60 & -5 \\
		4 & 25 & 30 & -5 \\
		5 & 50 & 50 & 0 \\
		6 & 80 & 70 & 10 \\										
		7 & 45 & 45 & 0 \\ \hline
		\textbf{Average} & \textbf{50} & \textbf{50} & \textbf{0} \\				
	\end{tabular}	
	\vspace{3mm}
	
	
	%\begin{split}
	\begin{equation}	
		ATE \equiv \frac{1}{N} \sum_{i=1}^{N} \tau_i
	\end{equation}
	\begin{equation}	
		 \frac{1}{N} \sum_{i=1}^{N} Y_i(1) - \frac{1}{N} \sum_{i=1}^{N} Y_i(0) = \frac{1}{N} \sum_{i=1}^{N} (Y_i(1)-Y_i(0)) = \frac{1}{N} \sum_{i=1}^{N} \tau_i
	\end{equation}	
	%\end{split}

		
\end{frame}

\begin{frame}{Repeated random sampling and expected values}

	Repeated random sampling
	\begin{itemize}
		\item Imagine we take an infinite number of samples of size N from the population
	\end{itemize}
	\vspace{3mm}
	Expected value
	\begin{itemize}
		\item Expected value of a variable is the average value of a random variable based on an infinite number of samples
		\item expected value of discrete random variable X: $E[X]=\sum xPr[X=x] $
		\begin{itemize}
		\item $ Pr[X=x] $ is probability that $X$ takes on the value $x$, where summation is taken over all possible values of X
		\end{itemize}			
		\item Example of expected value of dice role, $ X $:
		\begin{itemize}
			\item $E[X]=(1)(\frac{1}{6}) +(2)(\frac{1}{6})+(3)(\frac{1}{6})+(4)(\frac{1}{6})+(5)(\frac{1}{6})+(6)(\frac{1}{6})=3.5$
		\end{itemize}

	\end{itemize}	

\end{frame}

\begin{frame}{Conditional expectations}

	Conditional expectations refer to subgroup averages \\
	\vspace{3mm}
	
	Example: $Y_i$=income; $d_i$=internship (0,1); $Z_i$=GPA	
	\begin{itemize}

		\item $ E[Y_i|d_i=1]$
		\begin{itemize}
			\item Expected value of (observed) income, given that student got internship
		\end{itemize}		 
		\item $ E[Y_i|Z_i >3.5]$
		\begin{itemize}
			\item Expected value of (observed) income, given that college GPA was greater than 3.5
		\end{itemize}
		\item $ E[Y_i(1)|d_i=1]$
		\begin{itemize}
			\item Expected value of of treated potential outcome, given that treatment student received treatment
		\end{itemize}		
		\item $ E[Y_i(0)|d_i=1]$
		\begin{itemize}
			\item Expected value of untreated potential outcome, given that student did receive internship
		\end{itemize}
	\end{itemize}

\end{frame}

\begin{frame}[shrink=10]{Conditional expectations, potential outcomes, and random assignment}

	Random variable $ D_i $
	\begin{itemize}
		\item $ D_i $ is a variable whose value is randomly assigned (e.g., coin flip, or some random number generator)
		\item Value of $ D_i $ determines whether person $i$ receives treatment
	\end{itemize}
	\vspace{3mm}	
	Potential outcomes
	\begin{itemize}
		\item $ E[Y_i(1)|D_i=1]$
		\begin{itemize}
			\item Treated potential outcome, given $i$ assigned to treatment
			\item Observed?: Yes 
		\end{itemize}	
		\item $ E[Y_i(1)|D_i=0]$
		\begin{itemize}
			\item Treated potential outcome, given $i$ not assigned to treatment
			\item Observed?: No 			
		\end{itemize}		 
		\item $ E[Y_i(0)|D_i=1]$
		\begin{itemize}
			\item Untreated potential outcome, given $i$ assigned to treatment
			\item Observed?: No		
		\end{itemize}		 
		\item $ E[Y_i(0)|D_i=0]$
		\begin{itemize}
			\item Untreated potential outcome, given $i$ not assigned to treatment
			\item Observed?: Yes		
		\end{itemize}		 
			 
	\end{itemize}		
\end{frame}

\begin{frame}[shrink=10]{Random assignment and unbiased inference: why random assignment works}

	Imagine 3 people [in our sample of 7] randomly assigned to internship \\
	\vspace{3mm}	
	
	\begin{tabular}{ l r r r}
		\multirow{2}{*} & $ Y_i(1) $ & $ Y_i(0) $ & $ \tau_i $ \\ \textbf{$ i $} & \textbf{Treated} & \textbf{Untreated} & \textbf{Treatment effect} \\ \hline
		1 & 65 & \textbf{60} & ? \\
		2 & 30 & \textbf{35} & ? \\
		3 & \textbf{55} & 60 & ? \\
		4 & \textbf{25} & 30 & ? \\
		5 & 50 & \textbf{50} & ? \\
		6 & 80 & \textbf{70} & ? \\										
		7 & \textbf{45} & 45 & ? \\ \hline
		\textbf{Avg. (observed)} & \textbf{45} & \textbf{53.75} & \textbf{-8.75} \\
		\textbf{Avg. (potential)} & \textbf{50} & \textbf{50} & \textbf{0} \\
	\end{tabular}				 
	
\end{frame}


\begin{frame}[shrink=10]{Random assignment and unbiased inference: why random assignment works}

	Now imagine that start over (re-sample): randomly assign 3 people to receive internship \\ 
	
	\begin{tabular}{ l r r r}
		\multirow{2}{*} & $ Y_i(1) $ & $ Y_i(0) $ & $ \tau_i $ \\ \textbf{$ i $} & \textbf{Treated} & \textbf{Untreated} & \textbf{Treatment effect} \\ \hline
		1 & \textbf{65} & 60 & 5 \\
		2 & 30 & \textbf{35} & -5 \\
		3 & 55 & \textbf{60} & -5 \\
		4 & \textbf{25} & 30 & -5 \\
		5 & 50 & \textbf{50} & 0 \\
		6 & \textbf{80} & 70 & 10 \\										
		7 & 45 & \textbf{45} & 0 \\ \hline
		\textbf{Avg. (observed)} & \textbf{56.67} & \textbf{47.5} & \textbf{9.17} \\
		\textbf{Avg. (potential)} & \textbf{50} & \textbf{50} & \textbf{0} \\								
	\end{tabular}
	\vspace{3mm}
	
	If we re-sampled an infinite number of times, and calculated the average of the ``average observed treatment effect'' it would equal the ``average potential treatment effect''
		
	
\end{frame}

\begin{frame}[shrink=10]{Why random assignment works}

	\begin{itemize}
		
		%\item Since people randomly assigned to treatment, assignment to treatment $ (D_i=1) $ not related to the value of treated potential outcome ($Y_i(1)$) or untreated potential outcome ($Y_i(0)$)
		%\item for the group randomly assigned to treatment ($D_i=1$), the expected value of the treated potential outcomes $E[Y_i(1)|D_i=1]$ equals the expected value of treated potential outcome for all people in sample $E[Y_i(1)$
		\item Every person has same probability of getting treatment ($D_i=1$); therefore, expected treated potential outcome among treated people is same as expected outcome for all people in sample
		\begin{itemize}
			\item $E[Y_i(1)|D_i=1]=E[Y_i(1)]$
		\end{itemize}
		\item Every person has same probability of getting control  ($D_i=0$); therefore, expected untreated potential outcome among untreated people is same as expected outcome for all people in sample
		\begin{itemize}
			\item $E[Y_i(0)|D_i=0]=E[Y_i(0)]$
		\end{itemize}
		%\item Potential outcomes for the treated group are on avg. the same as potential outcomes of control group
		\item Because assignment to treatment is random:
		\begin{itemize}
			\item Assignment to treatment has no effect on value of the potential outcomes; it just affects which potential outcome is observed for each person
			\item Assignment to treatment has no relationship to characteristics (e.g., parental income) that affect value of potential outcomes
		\end{itemize}		
		\item $ATE=E[Y_i(1)-Y_i(0)]=  E[Y_i(1)] - E[Y_i(0)] = E[Y_i(1)|D_i=1]-E[Y_i(0)|D_i=0]$
		\begin{itemize}
			\item recall: for each person, only difference between $ Y_i(1) $ and $Y_i(0)$ is the treatment
		\end{itemize}
	\end{itemize}
		
	
\end{frame}


\begin{frame}[shrink=1]{Observed outcomes, self-select into internship}

	Imagine if people self-selected into the internship; do you think assignment to treatment would be unrelated to value of potential outcomes $ Y_i(1) $ and $Y_i(0)$ \\
	\vspace{3mm}	
	\begin{tabular}{ l r r r}
		\multirow{2}{*} & $ Y_i(1) $ & $ Y_i(0) $ & $ \tau_i $ \\ \textbf{$ i $} & \textbf{Treated} & \textbf{Untreated} & \textbf{Treatment effect} \\ \hline
		1 & \textbf{65} & 60 & ? \\
		2 & 30 & \textbf{35} & ? \\
		3 & \textbf{55} & 60 & ? \\
		4 & 25 & \textbf{30} & ? \\
		5 & \textbf{50} & 50 & ? \\
		6 & \textbf{80} & 70 & ? \\										
		7 & 45 & \textbf{45} & ? \\ \hline
		\textbf{Average} & \textbf{59.75} & \textbf{36.67} & \textbf{23.08} \\				
	\end{tabular}		
	\vspace{3mm}
	
	The same characteristics (e.g., parental income, GPA) that determine the value of the dependent variable (income) also drive selection into the treatment (internship)
\end{frame}

\section[Observational design]{Observational design}

\subsection[Regression]{Components of the Population Regression Model}

\begin{frame}{Population linear regression model}
	\begin{itemize}
	\item *Population* Linear Regression Model
		\begin{itemize}
		\item $Y_{i} = \beta_{0} + \beta_{1}X_{i} + u_{i}$
		\end{itemize}
	\item Where:
		\begin{itemize}
		\item $Y_{i}$ = income for person i 
		\item $ X_{i}$ = hours worked for person i 
		\item $ \beta_{0}$ (called ``population intercept") = average income for someone with X=0 (i.e., works zero hours)
		\item $ \beta_{1}$ (called ``population regression coefficient") = average effect of a one-unit increase in X on value of Y
		\item $u_{i}$ (called ``error terms") = all other variables not included in your model that affect value of Y
		\end{itemize}
	\vspace{3mm}
	\item Draw scatterplot and population regression line
		\begin{itemize}
			\item label components (e.g., residual=actual-predicted)
		\end{itemize}
	\end{itemize}
\end{frame}

\begin{comment}

\begin{frame}{Population linear regression model}
	\begin{itemize}
	\vspace{5mm}
	\item *Population* Linear Regression Model
		\begin{itemize}
		\item $Y_{i} = \beta_{0} + \beta_{1}X_{i} + u_{i}$
		\end{itemize}
	\vspace{5mm}
	\item Draw Picture
		\begin{itemize}
		\item Draw scatterplot of population
		\vspace{2mm}
		\item Draw population regression line
		\vspace{2mm}
		\item Label different components of population linear regression model
			\begin{itemize}
			\item Residual=actual=expected
			\end{itemize}
		\end{itemize}
	\end{itemize}
\end{frame} 


\begin{frame}{Population linear regression model}
	\begin{itemize}
	\item Population Linear Regression Model
		\begin{itemize}
		\item $Y_{i} = \beta_{0} + \beta_{1}X_{i} + u_{i}$
		\end{itemize}
	\item *population* linear regression model contains two ``population parameters" based on the data from entire population
		\begin{itemize}
		\item $ \beta_{0}$ (called ``population intercept") = average income for someone with X=0 (i.e., works zero hours)
		\item $ \beta_{1}$ (called ``population regression coefficient") = average effect of a one-unit increase in X on value of Y
		\end{itemize}
	\item Questions for class:
		\begin{itemize}
		\item How do we know that these are population parameters? 
		\item Do we usually know the value of $\beta_{0}$ and $\beta_{1}$?
		\end{itemize}
	\end{itemize}
\end{frame}

\end{comment}

\begin{frame}{Population regression coefficient, $\beta_{1}$}
	\footnotesize %too much text on this slide had to decrease font size for it all to fit and remove first set of bullets
	RQ: What is the effect of hours worked per week (X) on income (Y)?
		\begin{itemize}
		\item Answer: population regression coefficient, $\beta_{1}$
		\item Estimating $\beta_{1}$ is the fundamental goal of program evaluation research
		\end{itemize}
	What is the population regression coefficient, $\beta_{1}$?
		\begin{itemize}
		\item $\beta_{1}$ measures the average change in Y for a one-unit increase in X
		\item Think of $\beta_{1}$ as measuring the slope of a line
			\begin{itemize}
			\item $\beta_{1} = \frac{\Delta Y}{\Delta X} = \frac{\Delta (income)}{\Delta (hours worked)}$
			\item $Example = \frac{\$5,000 \Delta \text{in income}}{\text{1 hour} \Delta \text{in hours worked per week}} = \$5,000 = \beta_{1} $
			\end{itemize}
		\end{itemize}
	Interpretation
		\begin{itemize}
		\item General interpretation:
			\begin{itemize}
			\item On average, a one-unit increase in X is associated with a $\beta_{1}$ increase in the value of Y
			\end{itemize}
		\item Interpretation for our research question:
			\begin{itemize}
			\item On average, a one-hour increase in hours worked per week (X) is associated with a \$$\beta_{1}$ increase in annual income
			\end{itemize}
		\item Imagine that $\beta_{1}$=2,000; How do we interpret this? $\beta_{1}$=4,000?
		\end{itemize}
\end{frame}

\begin{comment}

\begin{frame}{Population regression coefficient, $\beta_{1}$}
	\begin{itemize}
	\item Other important stuff
		\begin{itemize}
		\vspace{5mm}
		\item If $\beta_{1}$ -the relationship between X and Y- is linear, then the average change in Y for a one-unit increase in X is the same, no matter what the starting value of X
			\begin{itemize}
			\item Draw in scatterplot
			\end{itemize}
		\vspace{5mm}
		\item $\beta_{1}$ is a *population parameter*; we don't know it. We estimate $\beta_{1}$ using sample data
		\vspace{5mm}
		\item $\beta_{1}$ measures the *average* effect on Y for one-unit increase in X; effect on an individual person may be different than this average effect
		\end{itemize}
	\end{itemize}
\end{frame}

\end{comment}

\begin{frame}{Population Intercept, $\beta_{0}$}
	\begin{itemize}
	\item RQ: What is the effect of hours worked per week (X) on annual income (Y) 
		\begin{itemize}
		\item $Y_{i} = \beta_{0} + \beta_{1}X_{i} + u_{i}$
		\end{itemize}
	\item $\beta_{0}$ (called ``population intercept")
		\begin{itemize}
		\item $\beta_{0}$ represents average value of Y when X=0
		\item In our example, $\beta_{0}$ is average income for someone who works zero hours per week (X=0)
		\item Draw in scatterplot
		\end{itemize}
	\item Note:
		\begin{itemize}
		\item Usually we are substantively interested in $\beta_{0}$
		\item Also, do not believe $\beta_{0}$ if there are few observations where X=0 (e.g., effect of height on income) 
		\end{itemize}
	\end{itemize}
\end{frame} 

\begin{comment}
\begin{frame}{Population linear regression \textbf{*line*}} 
	\begin{itemize}
	\item Population linear regression model
		\begin{itemize}
		\item $Y_{i} = \beta_{0} + \beta_{1}X_{i} + u_{i}$
		\end{itemize}
		\vspace{5mm}
	\item Population linear regression line
		\begin{itemize}
		\item $Y_{i} = \beta_{0} + \beta_{1}X_{i}$
		\vspace{5mm}
		\item Population linear regression line does not include the error term, $u_{i}$
		\vspace{5mm}
		\item It is just a linear line, like the scatterplot if the scatterplot contained all observations in the population [show]
		\vspace{5mm} 
		\item It measures the ``average" or ``expected" relationship between X and Y, ignoring variables that we excluded from the model
		\end{itemize}		
	\end{itemize}
\end{frame}


\begin{frame}[shrink=20]{Population linear regression \textbf{*line*}} %too much text, had to shrink here
	\begin{itemize}
	\item It measures the *expected* relationship between X and Y
		\begin{itemize}
		\item $Y_{i} = \beta_{0} + \beta_{1}X_{i}$ 
		\end{itemize}
		\vspace{5mm}
	\item Population regression line and Expected value, E(Y)
		\begin{itemize}
		\item Expected value of Y [one variable]
			\begin{itemize}
			\item $E(Y) = \mu_{Y}$
			\end{itemize}
		\item Expected value of Y, given the value of X [relationship between two vars]
			\begin{itemize}
			\item $E(Y | X) = \beta_{0} + \beta_{1}X_{i}$
			\item population regression line is expected value of Y for a given value of X
			\end{itemize}
		\end{itemize}
		\vspace{5mm}
	\item Population regression line and prediction
		\begin{itemize}
		\item If we know value of parameters $\beta_{0}$ and $\beta_{1}$, we can predict value of Y
		\item Example: assume, $\beta_{0}$ = 5,000, $\beta_{1}$ = 1,000, and $X_{i}$ = 40
			\begin{itemize}
			\item $E(Y | X) = \beta_{0} + \beta_{1}X_{i}$
			\item $E(Y | 40) = 5,000 + 1,000 * 40$
			\item Predict that someone who works 40 hours per week makes \$45,000 per year 
			\end{itemize}
		\item Note: Will spend time on prediction next week
		\end{itemize}
	\end{itemize}
\end{frame}

\end{comment}

\begin{frame}{Thinking about $u_{i}$ as the ``error term"}
	\begin{itemize}
	\item Population linear regression model
		\begin{itemize}
		\item $Y_{i} = \beta_{0} + \beta_{1}X_{i} + u_{i}$
		\item $Y$=income ; $X_{i}$ = hours worked
		\end{itemize}
	\item Thinking about $u_{i}$ as the ``error term"
		\begin{itemize}
		\item In econometrics:
			\begin{itemize}
			\item Error term, $u_{i}$, consists of all other variables not included in your model that affect the dependent variable
			\item This interpretation of $u_{i}$ is very important for program evaluation research
			\item What variables besides hours worked (X) affect income (Y)?
			\end{itemize}
		\item In ``conventional" statistics textbooks:
			\begin{itemize}
			\item Overall error in prediction of Y (due to random variation)
			\end{itemize}
		\end{itemize}
	\end{itemize}
\end{frame}


\begin{frame}{Thinking about $u_{i}$ as the ``residual"}
	\begin{itemize}
	\item Population linear regression model
		\begin{itemize}
		\item $Y_{i} = \beta_{0} + \beta_{1}X_{i} + u_{i}$
		\item $Y$=income ; $X_{i}$ = hours worked
		\end{itemize}
	\item Draw scatterplot
		\begin{itemize}
		\item Population regression line represents the predicted value of Y (income) for each value of X (hours worked)
		\end{itemize}	
	\item Thinking about $u_{i}$ in terms of each observation, i
		\begin{itemize}
		\item $Y_{i}$ = actual value of income for person i
		\item $(\beta_{0} + \beta_{1}X_{i})$ = population regression line
			\begin{itemize}
			\item Equals the predicted value of income for person i with hours worked = $X_{i}$
			\end{itemize}
		\item Residual, $u_{i}$
			\begin{itemize}
			\item Residual, $u_{i}$, is the difference between actual value, $Y_{i}$, and predicted value from the population regression model for observation i
			\item $u_{i} = Y_{i} - (\beta_{0} + \beta_{1}X_{i})$
			\end{itemize}
		\end{itemize}
	\end{itemize}
\end{frame}

\subsection[Estimation]{Estimating Regression Parameters}

\begin{frame}{General things we do in regression analysis}
	\begin{enumerate}
	\item Estimation
		\begin{itemize}
		\item Choose estimates of $\beta_{0}$ and $\beta_{1}$ based on sample data
		\item $\hat{\beta_{0}}$ is an estimate of $\beta_{0}$; $\hat{\beta_{1}}$ is an estimate of $\beta_{1}$ 
		%\item How should we draw a line through the data to estimate the population slope (i.e., to estimate %$\beta_{1}$)
		%	\begin{itemize}
		%	\item To draw a straight line we just need to know Y-intercept and slope
		%	\end{itemize}
		\end{itemize}
	\item Prediction
		\begin{itemize}
		\item What is the predicted value of Y for someone with a particular value of X
			\begin{itemize}
			\item e.g., what is the predicted income for someone w/ an undergraduate degree in chemistry?
			\end{itemize}
		\end{itemize}
	\item Hypothesis testing [focus of causal inference]
		\begin{itemize}
		\item  Causal interpretation of $\beta_{1}$:
		\begin{itemize}
			\item  the effect of a one-unit increase in $ X $ is a $\beta_{1}$ increase in $ Y $		
		\end{itemize}
		\item Hypothesis testing and confidence intervals about $\beta_{1}$
		\begin{itemize}
			\item  Use $\hat{\beta_{1}}$ to test hypotheses about $\beta_{1}$
			\item  If we knew $\beta_{1}$, we would not need hypothesis testing
		\end{itemize}


		\end{itemize}
	\end{enumerate}
\end{frame}

\begin{comment}
\begin{frame}{Step 1 of regression: estimate parameters}
	\begin{itemize}
	\item Population linear regression model
		\begin{itemize}
		\item $Y_{i} = \beta_{0} + \beta_{1}X_{i} + u_{i}$
		\end{itemize}
	\vspace{5mm}
	\item Goal of estimation
		\begin{itemize}
		\item use sample data to estimate population intercept, $\beta_{0}$, and population regression coefficient, $\beta_{1}$
		\item $\hat{\beta_{0}}$ is an estimate of $\beta_{0}$; $\hat{\beta_{1}}$ is an estimate of $\beta_{1}$ 
			\begin{itemize}
			\item how do we know that symbol $\hat{\beta_{1}}$ is based on sample data?
			\end{itemize}
		\end{itemize}	
	\vspace{5mm}
	\item Estimation problem:
		\begin{itemize}
		\item Need to develop a method for choosing values of $\hat{\beta_{0}}$ and $\hat{\beta_{1}}$
		\end{itemize}	
	\end{itemize}
\end{frame}


\begin{frame}{Estimation (population mean)}
	\begin{itemize}
	\item In intro stat class we faced a similar problem:
		\begin{itemize}
		\item Use sample to calculate ``best" estimate of population mean, $\mu_{Y}$
		\item We decided that sample mean, $\overline{Y}$, was the ``best" estimate of $\mu_{Y}$
		\end{itemize}	
	\item Criteria used to decide $\overline{Y}$ was the ``best" estimate of $\mu_{Y}$?:
		\begin{itemize}
		\item Imagine that \emph{m} represents potential estimates for $\mu_{Y}$
		\item Goal: choose the value, \emph{m}, that minimizes the ``sum of squares"
			\begin{itemize}
			\item Sum of squares = $\sum_{i=1}^{n} (Y_{i} - m)^{2}$
			\vspace{2mm}
			\item $\overline{Y}$ is the value of m that minimizes sum of squares 
			\vspace{2mm}
			\item So $\overline{Y}$ is the ``least squares" estimator of $\mu_{Y}$
			\end{itemize}
		\end{itemize}		
	\item Draw two scatterplots:
		\begin{itemize}
		\item (1) horizontal line representing sample mean; (2) horizontal line representing some other value (e.g., median)
		\item Show some calculations of sum of squared errors for each
		\end{itemize}
	\end{itemize}
\end{frame}
\end{comment}

\begin{frame}[shrink=10]{Estimation (regression) [SKIP]}
	\begin{itemize}
	\item Problem in regression:
		\begin{itemize}
		\item Need to develop a method for choosing values of $\hat{\beta_{0}}$ and $\hat{\beta_{1}}$ 
		\item Solution: similar to what we did for population mean 
		\end{itemize}
	\vspace{5mm}	
	\item First, some terminology (draw scatterplot):
		\begin{itemize}
		\vspace{2mm}	
		\item $Y_{i}$ is the actual value of Y for individual i 
		\vspace{2mm}	
		\item $\hat{Y_{i}}$ is the predicted value $Y_{i}$, based on sample data
			\begin{itemize}
			\item $\hat{Y_{i}} = \hat{\beta_{0}} + \hat{\beta_{1}}X_{i}$
			\end{itemize}
		\vspace{2mm}	
		\item Residual, $\hat u_{i}$ = difference between actual, $Y_{i}$, and predicted, $\hat{Y_{i}}$
			\begin{itemize}
			\item $Y_{i} - \hat{Y_{i}} = \hat u_{i}$
			\item $Y_{i} - (\hat{\beta_{0}} + \hat{\beta_{1}}X_{i}) = \hat u_{i}$
			\item Note: residuals, $\hat u_{i}$, are often referred to as ``errors"
			\end{itemize}
		\end{itemize}		
	\end{itemize}
\end{frame}


\begin{frame}{Estimation (regression)  [SKIP]}
	\begin{itemize}
	\item Criteria for choosing $\hat{\beta_{0}}$ and $\hat{\beta_{1}}$
		\begin{itemize}
		\item Choose values for that minimize ``sum of squared residuals"
		\end{itemize}
	\item Residuals, $\hat u_{i}$
		\begin{itemize}
		\item $\hat u_{i} = Y_{i} - \hat{Y_{i}} = Y_{i} - (\hat{\beta_{0}} + \hat{\beta_{1}}X_{i})$
		\end{itemize}
	\item Sum of squared residuals [or ``sum of squared errors"]:
		\begin{itemize}
		\item $\sum_{i=1}^{n} (Y_{i} - \hat{Y_{i}})^{2}$
		\vspace{2mm}
		\item $\sum_{i=1}^{n} (Y_{i} - (\hat{\beta_{0}} + \hat{\beta_{1}}X_{i}))^{2}$
		\vspace{2mm}
		\item $\sum_{i=1}^{n} (\hat u_{i})^{2}$
		\end{itemize}
	\item Draw on scatterplot with two different regression lines
	\end{itemize}
\end{frame}


\begin{frame}{Ordinary least squares estimates}
	\begin{itemize}
	\item The OLS estimates $\hat{\beta_{0}}$ and $\hat{\beta_{1}}$ are the values that minimize the sum of squared residuals:
		\begin{itemize}
		\item $\sum_{i=1}^{n} (Y_{i} - \hat{Y_{i}})^{2} = \sum_{i=1}^{n} (Y_{i} - (\hat{\beta_{0}} + \hat{\beta_{1}}X_{i}))^{2} = \sum_{i=1}^{n} (\hat u_{i})^{2}$
		\end{itemize}
		\vspace{2mm}		
	\item This minimization problem is solved using calculus
		\begin{itemize}
		\item Stata does this for you
		\end{itemize}
	\item Important point:
		\begin{itemize}
		\item Any other choice of $\hat{\beta_{0}}$ and $\hat{\beta_{1}}$ will result in higher sum of squared errors
			\begin{itemize}
			\item Same idea as when we found estimate of population mean, $\mu_{Y}$
			\end{itemize}
		\item Draw two scatterplots: one with OLS estimates; one with non-OLS estimates (e.g., mean)
		\end{itemize}
	\end{itemize}
\end{frame}


\begin{frame}{OLS prediction line}
	\begin{itemize}
	\item *Population* linear regression model
		\begin{itemize}
		\item $Y_{i} = \beta_{0} + \beta_{1}X_{i} + u_{i}$
		\item Where Y =  income (\$000); X= hours worked
		\end{itemize}
	\item OLS prediction line (based on OLS estimates)
		\begin{itemize}
		\item $\hat{Y_{i}} = \hat{\beta_{0}} + \hat{\beta_{1}}X_{i}$
		\item Note: OLS prediction line also called ``OLS regression line"
		\end{itemize}
	\item Imagine OLS estimates are $\hat{\beta_{0}}$=5 and $\hat{\beta_{1}}$=2
		\begin{itemize}
		\item What is the predicted income for someone who works 0 hours per week?
		\item What is the predicted income for someone who works 10 hours per week?
		\end{itemize}
	%\item Draw scatterplot; write out pop regression model; OLS prediction line; OLS prediction line w/ estimates
	\end{itemize}
\end{frame}


\begin{comment}
\begin{frame}{OLS prediction line and residuals}
	\begin{itemize}
	\item *Population* linear regression model
		\begin{itemize}
		\item $Y_{i} = \beta_{0} + \beta_{1}X_{i} + u_{i}$
		\end{itemize}
	\item OLS prediction line (based on OLS estimates)
		\begin{itemize}
		\item $\hat{Y_{i}} = \hat{\beta_{0}} + \hat{\beta_{1}}X_{i}$
		\end{itemize}
	\item Residuals, $\mu_{i}$
		\begin{itemize}
		\item Residual= (actual value -  predicted value based on OLS)
			\begin{itemize}
			\item Each observation has a residual
			\end{itemize}
		\item Residual= $\hat u_{i} = Y_{i} - \hat{Y_{i}} $
		\item Residual= $\hat u_{i} = Y_{i} - (\hat \beta_{0} + \hat \beta_{1}X_{i})$
		\end{itemize}
	\item Show residuals on scatterplot in Stata
		\begin{itemize}
		\item Prediction = mean vs. prediction = OLS prediction
		\end{itemize}
	\end{itemize}
\end{frame}

\end{comment}

\subsection[Assumption I]{Ordinary least squares (OLS) Assumption I}


\begin{frame}{OLS Assumption 1 (mathematically)}
	\begin{itemize}
	\item Role of assumptions in statistics and causal inference
	\item Population linear regression model
		\begin{itemize}
		\item $Y_{i} = \beta_{0} + \beta_{1}X_{i} + u_{i}$
			\begin{itemize}
			\item Y= HS test score; X=0/1 MAS; $u_{i}$ = all other variables that affect Y but were not included in regression model
			\end{itemize}
		\end{itemize}
	\item Assumption 1 mathematically
		\begin{itemize}
			\item $E(u_{i}|X_{i}) = 0$
			\item ``expected value of $u_{i}$, given any value of $X$, equals $0$''
		\end{itemize}
	\item OLS Assumption 1 in words
		\begin{itemize}
		\item the independent variable $X_{i}$ is unrelated to the ``other variables", $u_{i}$, not included in model
				
			\item Pretend that $u_{i}$ consists of only one variable (e.g., ``grit'')
			\item OLS assumption 1 states that the mean value of omitted variable is equal to zero no matter what the value of variable X is			
		\end{itemize}

	\end{itemize}
\end{frame}



\begin{frame}[shrink=10]{OLS Assumption 1}
	\begin{itemize}
	\item Population linear regression model
		\begin{itemize}
		\item $Y_{i} = \beta_{0} + \beta_{1}X_{i} + u_{i}$
			\begin{itemize}
			\item Y= HS test score; X=0/1 MAS; $u_{i}$ = all other variables that affect Y but were not included in regression model
			\end{itemize}
		\end{itemize}
	\item Assumption 1: $E(u_{i}|X_{i}) = 0$
		\begin{itemize}
		\item In words: the independent variable $X_{i}$ is unrelated to the ``other variables", $u_{i}$, not included in model
		\end{itemize}
	\item Assumption is *always* satisfied in random assignment experiment
		\begin{itemize}
		\item Example: effect of MAS participation (X) on graduation (Y)
			\begin{itemize}
			\item X=0 (non participant); X=1 (MAS participant) 
			\end{itemize}	
		\item We randomly assign students to MAS participation (X)
		\item Other factors, $u_{i}$ (includes ``grit", parental involvement, ``aptitude", etc.) are *by construction* unrelated to values of X because we randomly assigned students to values of X
		\end{itemize}
	\item In observational studies, this assumption is usually violated
		\begin{itemize}
		\item E.g., MAS participation (X) is likely correlated with omitted variables $u_{i}$, (e.g. motivation) that affect Y
		\end{itemize}	
	\end{itemize}
\end{frame}



\begin{frame}{OLS Assumption 1 in practice}
	\begin{itemize}
	\item Assumption 1: $E(u_{i}|X_{i}) = 0$
		\begin{itemize}
		\item In words: the independent variable $X_{i}$ is unrelated to the ``other factors", $u_{i}$, not included in model
		\end{itemize}
	\item How to think about it in practice:
		\begin{itemize}
		\item $Y_{i} = \beta_{0} + \beta_{1}X_{i} + u_{i}$
		\item Are there any variables that are not in your model that affect Y and have a relationship (positive or negative) with X? If so, Assumption 1 is violated
		\end{itemize}	
	\item Any omitted variables that violate assumption 1?
		\begin{itemize}
		\item Effect of participating in fraternity (X) on GPA (Y)?
		\item Effect of years of education (X) on income (Y)?
		\item Effect of participating in ``summer bridge program" (X) on first-year retention (Y)?
		\item Effect of participating in Think Tank (X) on first-year retention?
		\end{itemize}
	\end{itemize}
\end{frame}


\subsection[Omitted Variable Bias]{Omitted Variable Bias}

\begin{frame}{Omitted Variable Bias}
	\begin{itemize}
	\item $Y_{i} = \beta_{0} + \beta_{1}X_{i} + u_{i}$; Y=test score; X=class size
		\begin{itemize}
		\item We want to know the *causal effect* of X on Y
		\end{itemize}
	\item Omitted Variable Bias
		\begin{itemize}
		\item Bias in estimate $\hat{\beta}_{1}$ due to variables being omitted from the model (part of $u_i$ rather than included in model)
		\item Omitted variable bias is really about OLS assumption \#1
		\end{itemize}
	\vspace{3mm}	
	\item For omitted variable bias to occur, the omitted variable ``Z" must satisfy two conditions:
		\begin{enumerate}
		\item Z affects value of Y (i.e. Z is part of $u$); \textbf{*and*}
		\item Z has a relationship with X (e.g., correlation; $corr(Z,X) \neq 0$)
		\end{enumerate}
	\end{itemize}
\end{frame}




\begin{frame}{OLS Assumption 1 \& Omitted Variable Bias}
	\begin{itemize}
	\item OLS Assumption 1: $E(u_{i}|X_{i}) = 0$
		\begin{itemize}
		\item the independent variable $X_{i}$ is unrelated to the ``other factors," $u_{i}$, that affect Y and are not included in model
		\item assumption violated if there are omitted variables that affect Y that also have relationship with X
		\end{itemize}
	\item Omitted Variable Bias
		\begin{itemize}
		\item omitted variable bias: bias in estimate $\hat{\beta}_{1}$ due to variables being omitted from the model
		\item For omitted variable bias to occur, the omitted variable ``Z" must satisfy two conditions:
			\begin{enumerate}
			\item Z affects value of Y (i.e. Z is part of $u$); \textbf{*and*}
			\item Z has a relationship with X (e.g., correlation; $corr(Z,X) \neq 0$)
			\end{enumerate}
		\end{itemize}
	\end{itemize}
\end{frame}


\begin{frame}{Omitted variable bias in practice}
	\begin{itemize}
	\item $Y_{i} = \beta_{0} + \beta_{1}X_{i} + u_{i}$; 
		\begin{itemize}
		\item Y=test score; X=class size
		\item Z = \% of students in district with English as a second language (ESL) [omitted from model]
		\end{itemize}
	\item For omitted variable bias to occur, the omitted variable ``Z" must satisfy two conditions:
		\begin{enumerate}
		\item Z affects value of Y (i.e. Z is part of $u$); \textbf{*and*}
			\begin{itemize}
			\item Would ESL affect standardized test scores? Why?
			\end{itemize}
		\item Z has a relationship with X (e.g., correlation; $corr(Z,X) \neq 0$)
			\begin{itemize}
			\item Is ESL likely to be correlated with student-teacher ratio? Why?
			\end{itemize}		
		\end{enumerate}
	\item If *both* conditions satisfied, then omission from ESL from model results in $\hat{\beta}_{1}$ having omitted variable bias
	\end{itemize}
\end{frame}


\begin{frame}{Omitted variable bias in practice}
	\begin{itemize}
	\item $Y_{i} = \beta_{0} + \beta_{1}X_{i} + u_{i}$; 
		\begin{itemize}
		\item Y=test score; X=class size; Z= variable omitted from model
		\end{itemize}
	\item For omitted variable bias to occur, the omitted variable ``Z" must satisfy two conditions:
		\begin{enumerate}
		\item Z affects [causal] value of Y (i.e. Z is part of $u$); \textbf{*and*}
		\item Z has a relationship with X (e.g., correlation; $corr(Z,X) \neq 0$)
		\end{enumerate}
	\item Would omitting Z = ``time of day test administered" result in omitted variable bias?
		\begin{itemize}
		\item Does test-time affect Y? Is test-time correlated with student-teacher ratio?
		\end{itemize}
	\item Would omitting Z = ``number of desks in class" result in omitted variable bias?
		\begin{itemize}
		\item Does parking space per pupil affect Y? Is parking space per pupil correlated with student-teacher ratio?
		\end{itemize}
	\end{itemize}
\end{frame}	

\begin{frame}{How to check for omitted variable bias}
	\begin{itemize}
	\item Ask yourself: 
		\begin{itemize}
		\item Could omitted variable Z affect Y? 
		\item Could omitted variable Z have some relationship with X?
		\end{itemize}
	\item How researchers think about omitted variable bias in practice
		\begin{itemize}
		\item Rely on logical argument
		\item Rely on theory
		\item Rely on prior research
			\begin{itemize}
				\item e.g., past studies show that Z affects Y
				\item past studies of ``effect of X on Y'' control for Z
			\end{itemize}
		\end{itemize}
		\item descriptive statistics (e.g., correlations)	
		\begin{itemize}
			\item only works if you have a good measure of Z
		\end{itemize}
	\end{itemize}
\end{frame}
		
\begin{comment}
\begin{frame}{How to check for omitted variable bias}
	\begin{itemize}
	\item In practice, diagnostic tests not used as much as logical arguments/literature review
		\begin{itemize}
		\vspace{2mm}
		\item Correlation only picks up linear relationships; omitted variable bias includes non-linear relationships
		\vspace{2mm}
		\item Relationship between X and Z is about ``conditional relationship" after controlling for other covariates
		\vspace{2mm}
		\item Sometimes you don't have a good measure of omitted variable Z
		\end{itemize}
	\end{itemize}
\end{frame}

\end{comment}

\begin{frame}{When is omitted variable bias big/small}
	\begin{itemize}
	\item $Y_{i} = \beta_{0} + \beta_{1}X_{i} + u_{i}$
	\vspace{3mm}
	\item Imagine there is only one omitted variable, Z
	\vspace{3mm}
	\item Omitted variable bias is likely big when:
		\begin{itemize}
		\item The omitted variable, Z, has a big causal effect on Y
		\item The correlation between X and the omitted variable, Z, is strong
		\end{itemize}
	\vspace{3mm}
	\item Example:
		\begin{itemize}
		\item Y= earnings ; X= participation in internship; Z= parental income
		\end{itemize}
	\end{itemize}
\end{frame}



\begin{frame}[shrink=10]{Omitted Variable Bias Formula [SKIP]}
	\begin{itemize}
	\item $Y_{i} = \beta_{0} + \beta_{1}X_{i} + u_{i}$
	\item Omitted variable bias formula [assume u is one variable]
		\begin{itemize}
		\item $\hat{\beta}_{1}$   $\underrightarrow{p}$   $\beta_{1} + corr(X_{i},u_{i}) * \frac{\sigma_{u}}{\sigma_{x}} $
		\end{itemize}
	\item What do different components of formula mean?
		\begin{itemize}
		\item $\underrightarrow{p}$= ``approaches this value as sample size increases"
		\item $\sigma_{u}$ = standard deviation of omitted variable(s), $u$?
		\item $\sigma_{X}$ = standard deviation of X
		\end{itemize}
	\item Formula in words
		\begin{itemize}
		\item As sample size increases, the OLS estimate, $\hat{\beta}_{1}$, approaches the population regression coefficient $\beta_{1}$ + the correlation between X and $u$ times the standard deviation of $u$ divided by the standard deviation of X
		\end{itemize}
	\item What value do we want $\hat{\beta}_{1}$ to approach as sample size increases?
	\item Omitted variable bias is this part of formula: $corr(X_{i},u_{i}) * \frac{\sigma_{u}}{\sigma_{x}} $
		\begin{itemize}
		\item Omitted variable bias is high when:
			\begin{itemize}
			\item strong correlation between X and u [can see from formula]
			\item the omitted variable, u, has a big effect on Y
			\end{itemize}
		\end{itemize}
	\end{itemize}
\end{frame}

\begin{frame}{Omitted Variable Bias Formula [SKIP]}
	\begin{itemize}
	\item $\hat{\beta}_{1}$   $\underrightarrow{p}$   $\beta_{1} + corr(X_{i},u_{i}) * \frac{\sigma_{u}}{\sigma_{x}} $
	\item Consistency
		\begin{itemize}
		\item The point estimate approaches the population parameter as sample size increases (e.g. imagine if your sample is the entire population)
		\item $\hat{\beta}_{1}$ is inconsistent when there is omitted variable bias; $\hat{\beta}_{1}$ does not approach $\beta_{1}$ as sample size increases
		\end{itemize}
	\item $corr(X_{i},u_{i})$
		\begin{itemize}
		\item If $corr(X_{i},u_{i}) = 0$ then there is no bias; $\hat{\beta}_{1}$   $\underrightarrow{p}$   $\beta_{1}$
		\item The size of the omitted variable bias depends on the strength of the correlation between X and u
		\end{itemize}
	\end{itemize}
\end{frame}

\begin{frame}[shrink=10]{Upwards/Downwards Bias [SKIP]}
	\begin{itemize}
	\item $Y_{i} = \beta_{0} + \beta_{1}* MAS_{i} + u_{i}$
		\begin{itemize}
		\item Y= graduation; X=0/1 in MAS; X2= motivation; X3= household income
		\item $\beta_{1}$ = true causal effect of participation in MAS on graduation
		\end{itemize}	
	\item	Upwards bias: $\hat{\beta}_{1}$   $>$   $\beta_{1}$
		\begin{itemize}
		\item Estimate of the causal effect $\hat{\beta}_{1}$, is greater than true causal effect,  $\beta_{1}$
		\item Example: Omit Z1, student motivation
			\begin{itemize}
			\item motivation positively affects graduation; positive correlation w/ participation in MAS
			\item If we omit student motivation from model, our estimate $\hat{\beta}_{1}$ is partially picking up positive effect of motivation on graduation
			\end{itemize}
		\end{itemize}
	\item	Downwards bias: $\hat{\beta}_{1}$   $<$   $\beta_{1}$
		\begin{itemize}
		\item Estimate of the causal effect $\hat{\beta}_{1}$, is less than true causal effect,  $\beta_{1}$
		\item Example: Omit Z2, household income
			\begin{itemize}
			\item household income positively affects graduation; negative correlation w/ participation in MAS
			\item If we omit household income from model, our estimate $\hat{\beta}_{1}$ is partially picking up negative effect of being low-income on graduation (because low income students are more likely to participate in MAS)
			\end{itemize}
		\end{itemize}
	\end{itemize}
\end{frame}

\begin{frame}{How to deal with Omitted Variable Bias}
	\begin{itemize}
	\item Random assignment experiments
		\begin{itemize}
		\item The ``gold standard"
		\end{itemize}
	\item Attempt to ``recreate" experimental conditions
		\begin{itemize}
		\item Multiple regression, matching
			\begin{itemize}
			\item Include omitted variables in your model, so they are no longer omitted
			\item This is the purpose of regression; otherwise you can use ANOVA
			\end{itemize}
		\item ``quasi-experimental" techniques
			\begin{itemize}
			\item More advanced methods for recreating experimental conditions 
			\item e.g., regression discontinuity; instrumental variables
			\end{itemize}
		\end{itemize}
	\end{itemize}
\end{frame}


\begin{frame}{How to deal with Omitted Variable Bias}
	\begin{itemize}
	\item Research question: What is the effect of student teacher ration (X) on district average test scores (Y)?
		\begin{itemize}
		\item $Y_{i} = \beta_{0} + \beta_{1}X_{i} + u_{i}$
		\item Imagine that we have two omitted variables 
			\begin{itemize}
			\item Z1= pct English as a Second Language (ESL)
			\item Z2= average income in the district
			\end{itemize}
		\end{itemize}
	\item Multiple regression:
		\begin{itemize}
		\item Attempt to recreate experimental conditions by including ``omitted variables" in your model, so they are no longer omitted
			\begin{itemize}
			\item Once you include Z1 and Z2 in your model, they are called ``control" variables because they control for omitted variable bias; also called ``covariates"
			\end{itemize}
		\end{itemize}
	\item Do in Stata
	\end{itemize}
\end{frame}

\begin{frame}[shrink=10]{Why no control variables in experiments}
	\begin{itemize}
	\item $Y_{i} = \beta_{0} + \beta_{1}X_{i} + u_{i}$; Y=graduation; X: 0=no MAS, 1=MAS
		\begin{itemize}
		\item Omitted vars: Z1= student motivation; Z2= household income
		\end{itemize}
	\item Omitted variable bias conditions:
		\begin{enumerate}
		\item Z affects value of Y (i.e. Z is part of $u$); \textbf{*and*}
		\item Z has a relationship with X (e.g., correlation)
		\end{enumerate}
	\item Imagine students randomly assigned to MAS
		\begin{itemize}
		\item Z1 = student motivation
			\begin{itemize}
			\item[(1)] does student motivation affect HS graduation(Y)?
			\item[(2)] could student motivation be related (e.g., correlation) with value of X (MAS)?
			\end{itemize}
		\item Z2 = household income
			\begin{itemize}
			\item[(1)] does household income affect HS graduation(Y)?
			\item[(2)] could household income be related to value of X (MAS)?
			\end{itemize}
		\end{itemize}
	\item Randomization in treatment vs. control group
		\begin{itemize}
		\item Any differences between treatment and control group on factors that affect Y have no relationship w/ value of X (treatment)
		\end{itemize}
	\end{itemize}
\end{frame}



\begin{frame}{Why no control variables in experiments: Tennessee STAR Experiment}
	\begin{itemize}
	\item Where to get data
		\begin{itemize}
		\item \url{https://dataverse.harvard.edu/dataset.xhtml?persistentId=hdl:1902.1/10766} 
		\end{itemize}
	\item Overview of experiment
		\begin{itemize}
		\item ``The Student/Teacher Achievement Ratio (STAR) was a four- year longitudinal class-size study funded by the Tennessee General Assembly and conducted by the State Department of Education. Over 7,000 students in 79 schools were randomly assigned into one of three interventions: small class (13 to 17 students per teacher), regular class (22 to 25 students per teacher), and regular-with-aide class (22 to 25 students with a full-time teacher's aide)
		\vspace{2mm}
		\end{itemize}
	\item RQ: what is effect of class-size treatment (X) on first-grade math scores (Y)
		\begin{itemize}
		\item Do in Stata
		\end{itemize}
	
	\end{itemize}
\end{frame}


\begin{frame}[shrink=10]{Conditional Independence Assumption}
	\begin{itemize}
	\item Assume students choose to participate in MAS
	\item $Y_{i} = \beta_{0} + \beta_{1}X_{1i} + \beta_{2}X_{2i} + \beta_{3}X_{3i} + u_{i}$
		\begin{itemize}
		\item Y= graduation; X=0/1 in MAS; X2= motivation; X3= income
		\end{itemize}
	\item Conditional independence assumption
		\begin{itemize}
		\item Once we include control variables, there are no omitted variables, Z, that satisfy *both* of these two conditions
			\begin{itemize}
			\item[(1)] Z affects value of Y (i.e. Z is part of $u$); \textbf{*and*}
			\item[(2)] Z has a relationship with X (e.g., correlation)
			\end{itemize}
		\end{itemize}
	\item If the conditional independence assumption is true:
		\begin{itemize}
		\item Once we include relevant control variables, there are no omitted variables that affect Y and have a systemic relationship with X
		\item Main point: if we satisfy the conditional independence assumption through control variables, then multiple regression is just as good as random assignment experiment!
			\begin{itemize}
			\item In random assignment experiments, there are omitted variables that affect Y, but none of these omitted variables have a systemic relationship with X because X is randomly assigned
			\end{itemize}
		\end{itemize}
	\end{itemize}
\end{frame}

\subsection[Multiple Regression]{Introduction to Multiple Regression}


\begin{frame}[shrink=10]{Population Multiple Regression Model}
	\begin{itemize}
	\item $Y_{i}$ = $\beta_{0} + \beta_{1}X_{1i} + \beta_{2}X_{2i} +$ .... $+\beta_{k}X_{ki} +  u_{i}$
	\item Where:
		\begin{itemize}
		\item $Y_{i}$ = observation i of dependent variable
		\item $X_{1i}$ = observation i for the first regressor, $X_{1}$
		\item $X_{2i}$ = observation i for the second regressor, $X_{2}$
		\item $X_{ki}$ = observation i for the kth regressor, $X_{2}$
		\item $\beta_{1}$ = population average effect on Y for a one-unit increase in $X_{1}$
		\item $\beta_{2}$ = population average effect on Y for a one-unit increase in $X_{2}$
		\item $\beta_{k}$ = population average effect on Y for a one-unit increase in $X_{k}$
		\item $\beta_{0}$ = average value of Y when the value of all independent variables, $X_{1}, X_{2},...X_{k}$, are equal to zero
		\item $u_{i}$ = all other variables that *affect* the value of $Y_{i}$ but are not included in the model (i.e., not $X_{1}$ or $X_{2}$)
		\item k = refers to the number of independent variables in your model
			\begin{itemize}
			\item e.g., model where independent variables are age, education level, and income has k=3
			\end{itemize}
		\end{itemize}
	\end{itemize}
\end{frame}

\begin{frame}[shrink=10]{Multivariate Regression \& Program Evaluation}
	\begin{itemize}
	\item $Y_{i} = \beta_{0} + \beta_{1}X_{1i} + \beta_{2}X_{2i} + \beta_{3}X_{3i} + u_{i}$
	\item Program evaluation research [econometrics]
		\begin{itemize}
		\item We are only interested in estimating $\beta_{1}$ [causal effect of X1 on Y]
		\item The only reason we include other variables in the model beside X1 is to eliminate omitted variable bias
		\item Therefore, we include all control variables that satisfy *both* conditions of omitted variable bias:
		\item Once we include control variables, there are no omitted variables, Z, that satisfy *both* of these two conditions
			\begin{itemize}
			\item[(1)] Z affects value of Y (i.e. Z is part of $u$); \textbf{*and*}
			\item[(2)] Z has a relationship with X (e.g., correlation)
			\end{itemize}
		\end{itemize}
	\item Traditional social science statistics
		\begin{itemize}
		\item Purpose of multiple regression is to add new variable to your model (e.g. $X_{3}$) to see effect of variable $X_{3}$ on Y
		\item Can lead to sloppy research! If you don't get an ``interesting" result for $\hat{\beta}_{1}$, then focus on a variable with a more interesting coefficient (e.g. $X_{3}$)
		\end{itemize}
	\end{itemize}
\end{frame}


\begin{frame}[shrink=10]{What does ``holding constant" mean?}
	\begin{itemize}
	\item RQ: What is the relationship between years of education(X1) on income(Y), after controlling for years of work experience (X2)?
	\item ``Holding the value of X2 constant"
		\begin{itemize}
		\item Means to estimate the relationship between X1 and Y when we don't allow value of X2 to vary [partial derivative]
		\item Said different: relationship between education (X1) and income (Y) for applicants that have same years of experience (X2)
		\end{itemize}	
	\item General interpretation of $\beta_{1}$ (assuming causal relationship):
		\begin{itemize}
		\item The average effect of a one-unit increase in $X_{1}$ is a $\beta_{1}$ unit increase in Y, holding the value of $X_{2}$ constant 
		\end{itemize}
	\item Interpretation of $\beta_{1}$, applied to example
		\begin{itemize}
		\item The effect of having one additional year of education (X1) on income (Y), when we don't allow value ``years of experience" (X2) to change
		\item Said different: the effect of increasing years of education on income for people who have same years of experience
		\end{itemize}
	\end{itemize}
\end{frame}

\end{document}

