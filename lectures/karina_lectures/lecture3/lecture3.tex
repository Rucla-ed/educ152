% Options for packages loaded elsewhere
\PassOptionsToPackage{unicode}{hyperref}
\PassOptionsToPackage{hyphens}{url}
\PassOptionsToPackage{dvipsnames,svgnames*,x11names*}{xcolor}
%
\documentclass[
  8pt,
  ignorenonframetext,
  dvipsnames]{beamer}
\usepackage{pgfpages}
\setbeamertemplate{caption}[numbered]
\setbeamertemplate{caption label separator}{: }
\setbeamercolor{caption name}{fg=normal text.fg}
\beamertemplatenavigationsymbolsempty
% Prevent slide breaks in the middle of a paragraph
\widowpenalties 1 10000
\raggedbottom
\setbeamertemplate{part page}{
  \centering
  \begin{beamercolorbox}[sep=16pt,center]{part title}
    \usebeamerfont{part title}\insertpart\par
  \end{beamercolorbox}
}
\setbeamertemplate{section page}{
  \centering
  \begin{beamercolorbox}[sep=12pt,center]{part title}
    \usebeamerfont{section title}\insertsection\par
  \end{beamercolorbox}
}
\setbeamertemplate{subsection page}{
  \centering
  \begin{beamercolorbox}[sep=8pt,center]{part title}
    \usebeamerfont{subsection title}\insertsubsection\par
  \end{beamercolorbox}
}
\AtBeginPart{
  \frame{\partpage}
}
\AtBeginSection{
  \ifbibliography
  \else
    \frame{\sectionpage}
  \fi
}
\AtBeginSubsection{
  \frame{\subsectionpage}
}
\usepackage{lmodern}
\usepackage{amssymb,amsmath}
\usepackage{ifxetex,ifluatex}
\ifnum 0\ifxetex 1\fi\ifluatex 1\fi=0 % if pdftex
  \usepackage[T1]{fontenc}
  \usepackage[utf8]{inputenc}
  \usepackage{textcomp} % provide euro and other symbols
\else % if luatex or xetex
  \usepackage{unicode-math}
  \defaultfontfeatures{Scale=MatchLowercase}
  \defaultfontfeatures[\rmfamily]{Ligatures=TeX,Scale=1}
\fi
% Use upquote if available, for straight quotes in verbatim environments
\IfFileExists{upquote.sty}{\usepackage{upquote}}{}
\IfFileExists{microtype.sty}{% use microtype if available
  \usepackage[]{microtype}
  \UseMicrotypeSet[protrusion]{basicmath} % disable protrusion for tt fonts
}{}
\makeatletter
\@ifundefined{KOMAClassName}{% if non-KOMA class
  \IfFileExists{parskip.sty}{%
    \usepackage{parskip}
  }{% else
    \setlength{\parindent}{0pt}
    \setlength{\parskip}{6pt plus 2pt minus 1pt}}
}{% if KOMA class
  \KOMAoptions{parskip=half}}
\makeatother
\usepackage{xcolor}
\IfFileExists{xurl.sty}{\usepackage{xurl}}{} % add URL line breaks if available
\IfFileExists{bookmark.sty}{\usepackage{bookmark}}{\usepackage{hyperref}}
\hypersetup{
  pdftitle={Introduction to Multivariate Regression \& Econometrics},
  pdfauthor={Lecture 3},
  colorlinks=true,
  linkcolor=Maroon,
  filecolor=Maroon,
  citecolor=Blue,
  urlcolor=blue,
  pdfcreator={LaTeX via pandoc}}
\urlstyle{same} % disable monospaced font for URLs
\newif\ifbibliography
\usepackage{color}
\usepackage{fancyvrb}
\newcommand{\VerbBar}{|}
\newcommand{\VERB}{\Verb[commandchars=\\\{\}]}
\DefineVerbatimEnvironment{Highlighting}{Verbatim}{commandchars=\\\{\}}
% Add ',fontsize=\small' for more characters per line
\usepackage{framed}
\definecolor{shadecolor}{RGB}{248,248,248}
\newenvironment{Shaded}{\begin{snugshade}}{\end{snugshade}}
\newcommand{\AlertTok}[1]{\textcolor[rgb]{0.94,0.16,0.16}{#1}}
\newcommand{\AnnotationTok}[1]{\textcolor[rgb]{0.56,0.35,0.01}{\textbf{\textit{#1}}}}
\newcommand{\AttributeTok}[1]{\textcolor[rgb]{0.77,0.63,0.00}{#1}}
\newcommand{\BaseNTok}[1]{\textcolor[rgb]{0.00,0.00,0.81}{#1}}
\newcommand{\BuiltInTok}[1]{#1}
\newcommand{\CharTok}[1]{\textcolor[rgb]{0.31,0.60,0.02}{#1}}
\newcommand{\CommentTok}[1]{\textcolor[rgb]{0.56,0.35,0.01}{\textit{#1}}}
\newcommand{\CommentVarTok}[1]{\textcolor[rgb]{0.56,0.35,0.01}{\textbf{\textit{#1}}}}
\newcommand{\ConstantTok}[1]{\textcolor[rgb]{0.00,0.00,0.00}{#1}}
\newcommand{\ControlFlowTok}[1]{\textcolor[rgb]{0.13,0.29,0.53}{\textbf{#1}}}
\newcommand{\DataTypeTok}[1]{\textcolor[rgb]{0.13,0.29,0.53}{#1}}
\newcommand{\DecValTok}[1]{\textcolor[rgb]{0.00,0.00,0.81}{#1}}
\newcommand{\DocumentationTok}[1]{\textcolor[rgb]{0.56,0.35,0.01}{\textbf{\textit{#1}}}}
\newcommand{\ErrorTok}[1]{\textcolor[rgb]{0.64,0.00,0.00}{\textbf{#1}}}
\newcommand{\ExtensionTok}[1]{#1}
\newcommand{\FloatTok}[1]{\textcolor[rgb]{0.00,0.00,0.81}{#1}}
\newcommand{\FunctionTok}[1]{\textcolor[rgb]{0.00,0.00,0.00}{#1}}
\newcommand{\ImportTok}[1]{#1}
\newcommand{\InformationTok}[1]{\textcolor[rgb]{0.56,0.35,0.01}{\textbf{\textit{#1}}}}
\newcommand{\KeywordTok}[1]{\textcolor[rgb]{0.13,0.29,0.53}{\textbf{#1}}}
\newcommand{\NormalTok}[1]{#1}
\newcommand{\OperatorTok}[1]{\textcolor[rgb]{0.81,0.36,0.00}{\textbf{#1}}}
\newcommand{\OtherTok}[1]{\textcolor[rgb]{0.56,0.35,0.01}{#1}}
\newcommand{\PreprocessorTok}[1]{\textcolor[rgb]{0.56,0.35,0.01}{\textit{#1}}}
\newcommand{\RegionMarkerTok}[1]{#1}
\newcommand{\SpecialCharTok}[1]{\textcolor[rgb]{0.00,0.00,0.00}{#1}}
\newcommand{\SpecialStringTok}[1]{\textcolor[rgb]{0.31,0.60,0.02}{#1}}
\newcommand{\StringTok}[1]{\textcolor[rgb]{0.31,0.60,0.02}{#1}}
\newcommand{\VariableTok}[1]{\textcolor[rgb]{0.00,0.00,0.00}{#1}}
\newcommand{\VerbatimStringTok}[1]{\textcolor[rgb]{0.31,0.60,0.02}{#1}}
\newcommand{\WarningTok}[1]{\textcolor[rgb]{0.56,0.35,0.01}{\textbf{\textit{#1}}}}
\setlength{\emergencystretch}{3em} % prevent overfull lines
\providecommand{\tightlist}{%
  \setlength{\itemsep}{0pt}\setlength{\parskip}{0pt}}
\setcounter{secnumdepth}{-\maxdimen} % remove section numbering

%packages
\usepackage{graphicx}
\usepackage{rotating}
\usepackage{hyperref}

\usepackage{tikz} % used for text highlighting, amongst others
\usepackage{comment}

%title slide stuff
%\institute{Department of Education}
%\title{Managing and Manipulating Data Using R}

%
\setbeamertemplate{navigation symbols}{} % get rid of navigation icons:
\setbeamertemplate{footline}[page number]

%\setbeamertemplate{frametitle}{\thesection \hspace{0.2cm} \insertframetitle}
\setbeamertemplate{section in toc}[sections numbered]
%\setbeamertemplate{subsection in toc}[subsections numbered]
\setbeamertemplate{subsection in toc}{%
  \leavevmode\leftskip=3.2em\color{gray}\rlap{\hskip-2em\inserttocsectionnumber.\inserttocsubsectionnumber}\inserttocsubsection\par
}

%define colors
%\definecolor{uva_orange}{RGB}{216,141,42} % UVa orange (Rotunda orange)
\definecolor{mygray}{rgb}{0.95, 0.95, 0.95} % for highlighted text
	% grey is equal parts red, green, blue. higher values >> lighter grey
	%\definecolor{lightgraybo}{rgb}{0.83, 0.83, 0.83}

% new commands

%highlight text with very light grey
\newcommand*{\hlg}[1]{%
	\tikz[baseline=(X.base)] \node[rectangle, fill=mygray] (X) {#1};%
}
%, inner sep=0.3mm
%highlight text with very light grey and use font associated with code
\newcommand*{\hlgc}[1]{\texttt{\hlg{#1}}}

%modifying back ticks to add grey background
\let\OldTexttt\texttt
\renewcommand{\texttt}[1]{\OldTexttt{\hlg{#1}}}


\begin{comment}

% Font
\usepackage[defaultfam,light,tabular,lining]{montserrat}
\usepackage[T1]{fontenc}
\renewcommand*\oldstylenums[1]{{\fontfamily{Montserrat-TOsF}\selectfont #1}}

% Change color of boldface text to darkgray
\renewcommand{\textbf}[1]{{\color{darkgray}\bfseries\fontfamily{Montserrat-TOsF}#1}}

% Bullet points
\setbeamertemplate{itemize item}{\color{BlueViolet}$\circ$}
\setbeamertemplate{itemize subitem}{\color{BrickRed}$\triangleright$}
\setbeamertemplate{itemize subsubitem}{$-$}

% Reduce space before lists
%\addtobeamertemplate{itemize/enumerate body begin}{}{\vspace*{-8pt}}

\let\olditem\item
\renewcommand{\item}{%
  \olditem\vspace{4pt}
}

% decreasing space before and after level-2 bullet block
%\addtobeamertemplate{itemize/enumerate subbody begin}{}{\vspace*{-3pt}}
%\addtobeamertemplate{itemize/enumerate subbody end}{}{\vspace*{-3pt}}

% decreasing space before and after level-3 bullet block
%\addtobeamertemplate{itemize/enumerate subsubbody begin}{}{\vspace*{-2pt}}
%\addtobeamertemplate{itemize/enumerate subsubbody end}{}{\vspace*{-2pt}}

%Section numbering
\setbeamertemplate{section page}{%
    \begingroup
        \begin{beamercolorbox}[sep=10pt,center,rounded=true,shadow=true]{section title}
        \usebeamerfont{section title}\thesection~\insertsection\par
        \end{beamercolorbox}
    \endgroup
}

\setbeamertemplate{subsection page}{%
    \begingroup
        \begin{beamercolorbox}[sep=6pt,center,rounded=true,shadow=true]{subsection title}
        \usebeamerfont{subsection title}\thesection.\thesubsection~\insertsubsection\par
        \end{beamercolorbox}
    \endgroup
}

\end{comment}

\title{Introduction to Multivariate Regression \& Econometrics}
\subtitle{HED 612}
\author{Lecture 3}
\date{}

\begin{document}
\frame{\titlepage}

\begin{frame}
  \tableofcontents[hideallsubsections]
\end{frame}
\hypertarget{prep}{%
\section{Prep}\label{prep}}

\begin{frame}{Download Data and Open R Script}
\protect\hypertarget{download-data-and-open-r-script}{}

\emph{Download Data and Open R Script}

\begin{enumerate}
\tightlist
\item
  Create a new data folder called ``gss''

  \begin{itemize}
  \tightlist
  \item
    HED612\_S21 \textgreater\textgreater\textgreater{} data
    \textgreater\textgreater\textgreater{} gss
  \end{itemize}
\item
  Download the GSS 2018 dataset from D2L (under Datasets)

  \begin{itemize}
  \tightlist
  \item
    Place the ``GSS2018.RData'' dataset into the ``gss'' folder you
    created in the previous step
  \end{itemize}
\item
  Download the Lecture 3 PDF and R files for this week

  \begin{itemize}
  \tightlist
  \item
    Place all files in HED612\_S21
    \textgreater\textgreater\textgreater{} lectures
    \textgreater\textgreater\textgreater{} lecture3
  \end{itemize}
\item
  Open the RProject (should be in your main HED612\_S21 folder)
\item
  Once the RStudio window opens, open the Lecture 3 R script by clicking
  on:

  \begin{itemize}
  \tightlist
  \item
    file \textgreater\textgreater\textgreater{} open file\ldots{}
    \textgreater\textgreater\textgreater{} {[}navigate to lecture 3
    folder{]} \textgreater\textgreater\textgreater{} lecture3.R
  \end{itemize}
\end{enumerate}

\end{frame}

\begin{frame}{Today and Next Week}
\protect\hypertarget{today-and-next-week}{}

\textbf{Today}

\begin{itemize}
\tightlist
\item
  Review of statistics (bivariate)
\item
  Some more R Basics
\item
  Intro to ``Gold Standard'' for causal inference and bivariate
  regression
\item
  Class exercise
\end{itemize}

\medskip

\textbf{HW \& Reading}

\begin{itemize}
\tightlist
\item
  HW\#3 posted on D2L
\item
  Stock \& Watson Ch. 4 {[}will cover Chapter 4 next 2 weeks{]}
\end{itemize}

\medskip

\textbf{Next Week}

\begin{itemize}
\tightlist
\item
  Bivariate regression cont.
\end{itemize}

\end{frame}

\begin{frame}{COE Student R Group}
\protect\hypertarget{coe-student-r-group}{}

\begin{itemize}
\tightlist
\item
  HED students are starting an R ``working group''

  \begin{itemize}
  \tightlist
  \item
    Will meet regularly
  \item
    Troubleshoot errors
  \item
    Work collaboratively on class assignments (highly encouraged!)
  \item
    Learn new skills via tutorials
  \end{itemize}
\item
  I have committed to providing some guidance and will participate when
  I can
\item
  So far looks like Thursdays from 5-7 may work

  \begin{itemize}
  \tightlist
  \item
    May consider meeting every two weeks\ldots{}
  \item
    If you want to join but have different availability fill out Doodle
    poll
  \item
    \href{https://doodle.com/poll/n3wpgsxtauhttmd8?utm_source=poll\&utm_medium=link}{Doodle
    Poll}
  \end{itemize}
\end{itemize}

\end{frame}

\begin{frame}[fragile]{Homework Review}
\protect\hypertarget{homework-review}{}

\begin{itemize}
\tightlist
\item
  Common Questions for PS\#2 {[}review as a class{]}

  \begin{itemize}
  \tightlist
  \item
    Packages vs Libraries in R

    \begin{itemize}
    \tightlist
    \item
      Packages are collections of R functions, data, and compiled code
      in a well-defined format, created to add specific functionality.
      We only need to install packages in R once via
      \texttt{install.packages()}
    \item
      Libraries are the ``directories'' in R where the packages are
      stored; we need to ``load'' these libraries via \texttt{library()}
      each time we would like to use any of the functionality associated
      with the package
    \end{itemize}
  \item
    Rounding in R

    \begin{itemize}
    \tightlist
    \item
      It's okay if your hand calculations are slightly off from R
      answers due to rounding\ldots{}
    \item
      Why is it rounding to an integer?
    \item
      R is an \textbf{object oriented programming language}; which means
      it saves ``results'' into various objects which ``function''
      differently
    \item
      Piping creates a tibble object\ldots which will round to the
      nearest integer
    \item
      {[}Show solutions in R and D2L{]}
    \end{itemize}
  \end{itemize}
\item
  Questions or Concerns?
\end{itemize}

\end{frame}

\hypertarget{review-of-statistics}{%
\section{Review of Statistics}\label{review-of-statistics}}

\begin{frame}{Hypothesis testing regarding the mean(s)}
\protect\hypertarget{hypothesis-testing-regarding-the-means}{}

Many hypotheses we have about education research can be phrased as yes
or no questions:

\begin{itemize}
\tightlist
\item
  Do students in smaller class sizes get better grades?
\item
  Is there a difference in earnings between men and women college
  graduates?
\item
  Do wealthier public schools get more recruiting visits by colleges and
  universites than poorer schools?
\end{itemize}

\medskip

Intro to Stats:

\begin{itemize}
\tightlist
\item
  Hypothesis testing regarding the population mean
\item
  Hypothesis testing regarding two populations
\end{itemize}

\end{frame}

\begin{frame}{Hypothesis testing: population mean}
\protect\hypertarget{hypothesis-testing-population-mean}{}

\textbf{Null hypothesis}: \(H_0\): \(E(Y) = \mu_{Y,0}\)

\textbf{Alternative hypothesis}: \(H_1\): \(E(Y) \ne \mu_{Y,0}\)

\medskip

\begin{itemize}
\tightlist
\item
  Sample mean, \(\bar{Y}\) or \(E(Y)\), is rarely exactly equal to the
  hypothesized value, \(\mu_{Y,0}\), in any given sample

  \begin{itemize}
  \tightlist
  \item
    Because the true population in fact does not equal the hypothesized
    value (the null hypothesis is false) OR
  \item
    Because the true population does equal the hypothesized value but
    \(\bar{Y}\) differs from the hypothesized value because of random
    sampling variation
  \end{itemize}
\end{itemize}

\medskip

Solution: we test the null hypothesis accounting for sampling variation!

\begin{itemize}
\tightlist
\item
  Perform a t-test to check whether there is a significant difference
  between the population mean (which is equal to the mean of sample
  means from last week's sampling distribution) and the hypothesized
  value

  \begin{itemize}
  \tightlist
  \item
    Compute the standard error of \(\bar{Y}\);
    \(SE(\bar{Y}) =s_Y/\sqrt{n}\)
  \item
    Compute the t-statistic; \(t = \bar{Y} - \mu_{Y,0}/SE(\bar{Y})\)\\
  \item
    Compute the p-value (significance probability) of null hypothesis;
    \(p-value = 2\Phi(-|t|)\)
  \item
    P-value is the probability of drawing \(\bar{Y}\) at least as far in
    the tails of its distribution under the assumption that the null
    hypothesis is correct as the sample average you actually computed
  \end{itemize}
\end{itemize}

\end{frame}

\begin{frame}{Hypothesis testing: population mean {[}Stock \& Watson
Example{]}}
\protect\hypertarget{hypothesis-testing-population-mean-stock-watson-example}{}

Example: In the population, college graduates earn \$20 an hour in the
labor market

\begin{itemize}
\tightlist
\item
  \textbf{Null hypothesis}: \(H_0\): \(E(Y) = \mu_{Y,20}\)
\item
  \textbf{Alternative hypothesis}: \(H_1\): \(E(Y) \ne \mu_{Y,20}\)
\item
  Random Sample of 200 college graduates

  \begin{itemize}
  \tightlist
  \item
    \(\bar{Y} = \$22.64\) is the mean hourly earnings of the sample
  \item
    \(s_Y = \$18.14\) is the standard deviation
  \end{itemize}
\end{itemize}

\medskip

\begin{itemize}
\tightlist
\item
  Step 1: Compute the \textbf{standard error of \(\bar{Y}\)}

  \begin{itemize}
  \tightlist
  \item
    \(SE(\bar{Y}) =s_Y/\sqrt{n}\)
  \end{itemize}
\item
  Step 2: Compute the \textbf{T-statistic}: measures the size (in units
  of standard error) of the difference between the population mean and
  the hypothesized value relative to the variation in the sample data

  \begin{itemize}
  \tightlist
  \item
    \(t = \bar{Y} - \mu_{Y,0}/SE(\bar{Y})\)
  \end{itemize}
\item
  Step 3: Compute the \textbf{P-value}: the probability of observing
  another \(\bar{Y}\) at least as different from \$20 as \$22.24 by pure
  random sampling variation assuming the null is correct

  \begin{itemize}
  \tightlist
  \item
    \(p-value = 2\Phi(-|t|)\)
  \item
    If the p-value is 0.039, then there is only a 3.9\% probability that
    a similar (\(\bar{Y} = \$22.24\)) would have been drawn if the null
    is true, we can reject the null hypothesis
  \item
    If the p-value is 0.40 (hypothetical), then there is about a 40\%
    probability that a similar (\(\bar{Y} = \$22.24\)) would have been
    drawn if the null is true; we cannot reject the null
  \end{itemize}
\end{itemize}

\end{frame}

\begin{frame}{Hypothesis testing: comparing means from two populations}
\protect\hypertarget{hypothesis-testing-comparing-means-from-two-populations}{}

Do men and women college graduates, on average, earn the same amount?

\textbf{Null hypothesis}: \(H_0\): \(\mu_{m} - \mu_{w} = 0\)

\textbf{Alternative hypothesis}: \(H_1\): \(\mu_{m} - \mu_{w} \ne 0\)

\medskip

\begin{itemize}
\tightlist
\item
  \$21.99 is the average hourly earnings of men in the sample of college
  graduates
\item
  \$18.48 is the average hourly earnings of men in the sample of college
  graduates
\item
  \$21.99 - \$18.48 = \$3.52
\end{itemize}

\medskip

\textbf{T-statistic} measures the size (in SE) of the difference between
the population wage gap and the hypothesized wage gap relative to the
variation in the sample data

\begin{itemize}
\tightlist
\item
  the greater the magnitude of t-statistic, the greater evidence against
  the null
\item
  the closer the magnitude of t-statistic to zero, the more likely it is
  that there is no signifcant difference between the population mean and
  hypothesized value
\end{itemize}

\medskip

The \textbf{P-value} is the probability of observing a difference of
\(\mu_{m} - \mu_{w}\) at least as different from zero as the observed
difference of \$3.52 by pure random sampling variation

\begin{itemize}
\tightlist
\item
  If the p-value is 0.05, then there is about a 5\% probability that a
  similar sample mean difference between men and women would have been
  drawn if the null hypothesis is true, so we can reject the null
  hypothesis
\item
  If the p-value is 0.40, then there is about a 40\% probability that
  the \$3.52 difference in sample average earnings between men and women
  could have arisen just by random sampling variation if the null
  hypothesis is true, so we would not reject the null hypothesis
\end{itemize}

\end{frame}

\begin{frame}{Relationships between two continuous variables}
\protect\hypertarget{relationships-between-two-continuous-variables}{}

Postive relationship, negative relationship, and no relationships

\medskip

\begin{enumerate}
\tightlist
\item
  Relationship between X and Y is positive
\end{enumerate}

\begin{itemize}
\tightlist
\item
  when X is ``high'', Y tend to be ``high''
\item
  when X is ``low'', Y tends to be ``low''
\item
  e.g., number of hours studying and GPA
\end{itemize}

\medskip

\begin{enumerate}
\setcounter{enumi}{1}
\tightlist
\item
  Relationship between X and Y is negative
\end{enumerate}

\begin{itemize}
\tightlist
\item
  when X is ``high'', Y tend to be ``low''
\item
  when X is ``low'', Y tends to be ``high''
\item
  e.g., number of school abscences and GPA
\end{itemize}

\medskip

\begin{enumerate}
\setcounter{enumi}{2}
\tightlist
\item
  No relationship between X and Y
\end{enumerate}

\begin{itemize}
\tightlist
\item
  knowing the value of X gives you does not tell you much about the
  value of Y
\item
  e.g., amount of ice cream consumed and GPA
\end{itemize}

\end{frame}

\begin{frame}{Today's Example}
\protect\hypertarget{todays-example}{}

Using the General Social Survey 2018 {[}sidebar: Data Goals for this
Class{]}:

\begin{itemize}
\tightlist
\item
  Nationally representative survey of adults in the United States
  conducted since 1972
\item
  Collects data on contemporary American society in order to monitor and
  explain trends in opinions, attitudes and behaviors

  \begin{itemize}
  \tightlist
  \item
    Demographic, behavioral, and attitudinal questions
  \item
    Covers topics like civil liberties, crime and violence, intergroup
    tolerance, morality, national spending priorities, psychological
    well-being, social mobility, etc.
  \end{itemize}
\item
  \href{https://gss.norc.org/}{GSS Website Link}
\end{itemize}

\medskip

We have two variables X and Y:

\begin{itemize}
\tightlist
\item
  X (independent variable) = hours worked per week
\item
  Y (dependent variable) = income
\end{itemize}

\medskip

Ways to investigate this relationship between X and Y:

\begin{itemize}
\tightlist
\item
  Graphically: scatterplots
\item
  Numerically: covariance (less used), correlation
\end{itemize}

\end{frame}

\begin{frame}[fragile]{Scatterplots}
\protect\hypertarget{scatterplots}{}

\begin{itemize}
\tightlist
\item
  Scatterplots will plot individual observations on an X and Y axis
\end{itemize}

\medskip

\begin{itemize}
\tightlist
\item
  Draw scatterplot of X (hours worked) and Y (income) by hand

  \begin{itemize}
  \tightlist
  \item
    Add a prediction line
  \end{itemize}
\end{itemize}

\medskip

\begin{itemize}
\tightlist
\item
  Residual

  \begin{itemize}
  \tightlist
  \item
    Difference between actual observed value of Y and predicted value of
    Y (given X)
  \end{itemize}
\end{itemize}

\medskip

\begin{itemize}
\tightlist
\item
  Relationship between X and Y is not perfect!
\end{itemize}

\medskip

Generate a scatteplot using \texttt{ggplot} in R script

\end{frame}

\begin{frame}{Covariance}
\protect\hypertarget{covariance}{}

Covariance measures the extent to which two variables move
together\ldots.

\begin{itemize}
\tightlist
\item
  If income is ``high'' when hours is worked is ``high'', then
  covariance is positive
\item
  If income is ``low'' when hours is worked is ``high'', then covariance
  is negative
\end{itemize}

\medskip

Population covariance, cov(X, Y)

\begin{itemize}
\tightlist
\item
  As with all population parameters, we don't know this!
\end{itemize}

\medskip

Population covariance, \(s_{XY}\) or \(\hat{\sigma}_{XY}\)

\begin{itemize}
\item
  Estimator of population covariance
\item
  \(s_{XY}= \hat{\sigma}_{XY}=\frac{\sum_{i=1}^{n}\left(X_{i}-\bar{X} \right) \left( Y_{i}-\bar{Y} \right)}{n-1}\)
\end{itemize}

\end{frame}

\begin{frame}{Sample Covariance}
\protect\hypertarget{sample-covariance}{}

\textbf{Formula}
\(s_{XY}= \hat{\sigma}_{XY}=\frac{\sum_{i=1}^{n}\left(X_{i}-\bar{X} \right) \left( Y_{i}-\bar{Y} \right)}{n-1}\)

\medskip

Example: Imagine we have 20 obs; \(\bar{X}=40; \bar{Y}=30\)

\medskip

Observation 1: \(X_1=50; Y_1=60\)

\begin{itemize}
\tightlist
\item
  \((X_{i}-\bar{X})( Y_{i}-\bar{Y})=(50-40)(60-30)=10*30=300\)
\item
  \(X_i > \bar{X}\) and \(Y_i > \bar{Y}\); so
  \((X_{i}-\bar{X})( Y_{i}-\bar{Y})\) is positive
\end{itemize}

Observation 2: \(X_1=45; Y_1=25\) -
\((X_{i}-\bar{X})( Y_{i}-\bar{Y})=(45-40)(25-30)=5*-5=-25\) -
\(X_i > \bar{X}\) and \(Y_i < \bar{Y}\); so
\((X_{i}-\bar{X})( Y_{i}-\bar{Y})\) is positive

\medskip

\(s_{XY}\) is the sum of these 20 calculations divided by 19 (n-1)

\end{frame}

\begin{frame}{Sample Covariance cont.}
\protect\hypertarget{sample-covariance-cont.}{}

\(s_{XY}\) is positive when X and Y move in the same direction

\begin{itemize}
\tightlist
\item
  \(X_i > \bar{X}\) usually coupled with \(Y_i > \bar{Y}\)
\item
  \(X_i < \bar{X}\) usually coupled with \(Y_i < \bar{Y}\)
\end{itemize}

\medskip

\(s_{XY}\) is negative when X and Y move in the same direction

\begin{itemize}
\tightlist
\item
  \(X_i > \bar{X}\) usually coupled with \(Y_i < \bar{Y}\)
\item
  \(X_i < \bar{X}\) usually coupled with \(Y_i > \bar{Y}\)
\end{itemize}

\end{frame}

\begin{frame}{Correlation}
\protect\hypertarget{correlation}{}

Problem with sample covariance, \(s_{XY}\)

\begin{itemize}
\tightlist
\item
  Covariance (like variance) dependes on the units of measurement
\item
  We can't compare the covaraince of X and Y vs covariance of X and Z
\end{itemize}

\medskip

Sample Correlation of Z and Y, \(r_{XY}\)

\begin{itemize}
\tightlist
\item
  Unitless measure of relationship between X and Y
\item
  Equals sample covariance, \(s_{XY}\), divided by the product of their
  individual sample standard deviations
\end{itemize}

\medskip

\textbf{Sample Correlation Formula}
\(r_{XY}=\frac{s_{XY}}{s_X*s_Y} = \frac{\hat{\sigma}_{XY}}{\hat{\sigma}_X \hat{\sigma}_Y}\)

\end{frame}

\begin{frame}{Sample Correlation}
\protect\hypertarget{sample-correlation}{}

\(r_{XY}=\frac{s_{XY}}{s_X*s_Y} = \frac{\hat{\sigma}_{XY}}{\hat{\sigma}_X \hat{\sigma}_Y}\)

\medskip

Correlations result in measures between -1 and 1

\medskip

``Type'' of relationship

\begin{itemize}
\tightlist
\item
  \(r_{XY}\) = 0 means there is no relationship between X and Y
\item
  \(r_{XY}\) \textgreater{} 0 means a positive correlation between X and
  Y

  \begin{itemize}
  \tightlist
  \item
    in other words, the variables move together
  \end{itemize}
\item
  \(r_{XY}\) \textless{} 0 means a negative correlation

  \begin{itemize}
  \tightlist
  \item
    in other words, the variables move in opposite directions
  \end{itemize}
\end{itemize}

\medskip

``Strength'' of relationship

\begin{itemize}
\tightlist
\item
  \(r_{XY}\) = \textbar0.1\textbar{} to \textbar0.3\textbar{} = weak
  relationship
\item
  \(r_{XY}\) = \textbar0.3\textbar{} to \textbar0.6\textbar{} = moderate
  relationship
\item
  \(r_{XY}\) = \textbar0.6\textbar{} to \textbar1\textbar{} = strong
  relationship
\end{itemize}

\medskip

Calculate correlations in R\ldots{}

\end{frame}

\begin{frame}{Linear vs Non-Linear Relationships}
\protect\hypertarget{linear-vs-non-linear-relationships}{}

Problem with covariance and correlation:

\begin{itemize}
\tightlist
\item
  Both measure linear relationships; these measures do not detect
  non-linear relationships
\end{itemize}

\medskip

See R script:

\begin{itemize}
\tightlist
\item
  Run correlation of income and age for respondents that identified as
  Black
\item
  Run Scatterplots
\end{itemize}

\end{frame}

\begin{frame}[fragile]{Random Assignment}
\protect\hypertarget{random-assignment}{}

Create the variable \texttt{randomvar} where values will be randomly
assigned

\begin{itemize}
\tightlist
\item
  Will randomly assign observations to values of 0 to 1000 for this new
  random var
\item
  Run correlation between \texttt{randomvar} and income

  \begin{itemize}
  \tightlist
  \item
    What is the relationship beween\texttt{randomvar} and income
    (positive, negative, no correlation)? Why?
  \end{itemize}
\end{itemize}

\medskip

This is the intuition behind \textbf{Randomized Control Trials} as the
``gold standard'' of progam evaluation research

\medskip

Randomization process

\begin{itemize}
\tightlist
\item
  Reseachers use a ``coin flip'' to sort participants (of a program,
  policy, treatment, etc.) into two groups: treatment group or control
  group

  \begin{itemize}
  \tightlist
  \item
    Coin flip = randomization
  \item
    Treatment is completely unrelated to participant's
    background/demographic characteristics
  \end{itemize}
\item
  \textbf{On average}, the two groups should be \emph{identical} in
  every way
\item
  Treat the treatment group; control group does not get treatment
\item
  Compare outcomes for the treatment and control group

  \begin{itemize}
  \tightlist
  \item
    \(\bar{Y}_{treat} - \bar{Y}_{control}\) = treatment effect
  \item
    examples: baby aspirin trial, Headstart, Tenessee Star Project
  \end{itemize}
\end{itemize}

\end{frame}

\begin{frame}[fragile]{In-Class Exercise {[}in groups{]}}
\protect\hypertarget{in-class-exercise-in-groups}{}

\begin{enumerate}
\tightlist
\item
  What is the correlation between how presitigious father's occupation
  is (\texttt{papres105plus}) and income (\texttt{incomev2})?
\item
  What is the correlation between hours worked (\texttt{hrs1}) and
  income (\texttt{incomev2})for respondents born in the U.S
  (\texttt{born})? How about for those not born in the U.S.? Do these
  correlations differ by birth in the U.S.?
\end{enumerate}

\begin{itemize}
\tightlist
\item
  Hint: If you run
  \texttt{gss\ \%\textgreater{}\%\ select(born)\ \%\textgreater{}\%\ val\_labels()}
  you'll see that the value \texttt{1} is ``yes'' and value \texttt{2}
  is ``no.''
\item
  Use tidyverse approach to run correlation; use the \texttt{filter()}
  function to run a correlation for each group
\end{itemize}

\begin{enumerate}
\tightlist
\item
  What is the correlation between hours worked (\texttt{hrs1}) and
  income (\texttt{incomev2}) for respondents with at least a bachelors
  degree? How about for those with less than a bachelors degree?
\end{enumerate}

\begin{itemize}
\tightlist
\item
  Hint: If you run
  \texttt{gss\ \%\textgreater{}\%\ select(degree)\ \%\textgreater{}\%\ val\_labels()}
  you'll see that the value \texttt{1} is ``high school'', the value
  \texttt{2} is ``junior college'', value \texttt{3} is ``bachelor'',
  and value \texttt{4} is ``graduate''.
\item
  Use tidyverse approach to run correlation; use the \texttt{filter()}
  function to run a correlation for those those with (at least) a
  bachelors and those without.
\end{itemize}

Code solutions on next slide\ldots{}

\end{frame}

\begin{frame}[fragile]{In-Class Exercise {[}Solutions{]}}
\protect\hypertarget{in-class-exercise-solutions}{}

\begin{Shaded}
\begin{Highlighting}[]
\CommentTok{#1 }
\NormalTok{gss }\OperatorTok\StringTok{ }\KeywordTok{summarise}\NormalTok{(}\KeywordTok{cor}\NormalTok{(papres105plus, incomev2, }\DataTypeTok{use =} \StringTok{"complete.obs"}\NormalTok{)) }

\CommentTok{#2 }
\NormalTok{gss }\OperatorTok\StringTok{ }\KeywordTok{select}\NormalTok{(born) }\OperatorTok\StringTok{ }\KeywordTok{val_labels}\NormalTok{()}
  
\NormalTok{gss }\OperatorTok\StringTok{ }\KeywordTok{filter}\NormalTok{(born}\OperatorTok{==}\DecValTok{1}\NormalTok{) }\OperatorTok
\StringTok{   }\KeywordTok{summarise}\NormalTok{(}\KeywordTok{cor}\NormalTok{(hrs1, incomev2, }\DataTypeTok{use =} \StringTok{"complete.obs"}\NormalTok{)) }
    
\NormalTok{gss }\OperatorTok\StringTok{ }\KeywordTok{filter}\NormalTok{(born}\OperatorTok{==}\DecValTok{2}\NormalTok{) }\OperatorTok
\StringTok{   }\KeywordTok{summarise}\NormalTok{(}\KeywordTok{cor}\NormalTok{(hrs1, incomev2, }\DataTypeTok{use =} \StringTok{"complete.obs"}\NormalTok{)) }

\CommentTok{#3}
\NormalTok{gss }\OperatorTok\StringTok{ }\KeywordTok{select}\NormalTok{(degree) }\OperatorTok\StringTok{ }\KeywordTok{val_labels}\NormalTok{()}
  
\NormalTok{gss }\OperatorTok\StringTok{ }\KeywordTok{filter}\NormalTok{(degree}\OperatorTok{>=}\DecValTok{3}\NormalTok{) }\OperatorTok
\StringTok{      }\KeywordTok{summarise}\NormalTok{(}\KeywordTok{cor}\NormalTok{(hrs1, incomev2, }\DataTypeTok{use =} \StringTok{"complete.obs"}\NormalTok{)) }
    
\NormalTok{gss }\OperatorTok\StringTok{ }\KeywordTok{filter}\NormalTok{(degree}\OperatorTok{<=}\DecValTok{2}\NormalTok{) }\OperatorTok
\StringTok{      }\KeywordTok{summarise}\NormalTok{(}\KeywordTok{cor}\NormalTok{(hrs1, incomev2, }\DataTypeTok{use =} \StringTok{"complete.obs"}\NormalTok{)) }
\end{Highlighting}
\end{Shaded}

\end{frame}

\end{document}
